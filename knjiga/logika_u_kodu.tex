\documentclass[11pt,a4paper,twoside,openright]{book}

% ===================================
% PAKETI
% ===================================

% Osnovni paketi za hrvatski jezik
\usepackage[utf8]{inputenc}
\usepackage[T1]{fontenc}
\usepackage{lmodern}
\usepackage[croatian]{babel}

% Matematika i simboli
\usepackage{amsmath,amssymb,amsthm}
\usepackage{mathtools}
\usepackage{stmaryrd}
\usepackage{bussproofs}

% Programski kod
\usepackage{listings}
\usepackage{minted}
\usepackage{tcolorbox}
\tcbuselibrary{minted,skins,breakable}

% Define the background color for code blocks
\definecolor{bg}{rgb}{0.95,0.95,0.95}
\setminted{breaklines,autogobble,linenos,numbersep=5pt}

% Define codeoutput environment
%\newenvironment{codeoutput}{%
%  \begin{tcolorbox}[title=Output,colback=gray!5!white,colframe=gray!75!black]%
%  \begin{verbatim}%
%}{%
%  \end{verbatim}%
%  \end{tcolorbox}%
%}
\newtcolorbox{codeoutput}{
    colback=gray!10,
    colframe=gray!50,
    breakable,
    fonttitle=\bfseries,
    title=Izlaz:,
    before upper=\small\ttfamily,
    sharp corners,
    boxrule=1pt,
    left=2mm,
    right=2mm,
    top=2mm,
    bottom=2mm
}

% Grafika i boje
\usepackage{graphicx}
\usepackage{tikz}
\usetikzlibrary{trees,arrows,positioning,calc,shapes,patterns,shadows.blur}
\usepackage{xcolor}
\usepackage{colortbl}

% Dizajn stranice
\usepackage[top=2.5cm, bottom=2.5cm, left=3cm, right=2.5cm]{geometry}
\usepackage{fancyhdr}
\usepackage{titlesec}
\usepackage{microtype}

% Hiperveze i reference
\usepackage[colorlinks=true,linkcolor=blue!70!black,citecolor=green!50!black,urlcolor=blue!70!black]{hyperref}
\usepackage{cleveref}

% Tablice i liste
\usepackage{booktabs}
\usepackage{array}
\usepackage{enumitem}
\usepackage{multicol}

% Dodatni paketi
\usepackage{epigraph}
\usepackage{marginnote}
\usepackage{mdframed}
\usepackage{caption}
\usepackage{subcaption}
\usepackage{newunicodechar}


% ===================================
% DEFINICIJE BOJA
% ===================================

\definecolor{pythonblue}{RGB}{53,114,165}
\definecolor{pythonyellow}{RGB}{255,217,64}
\definecolor{pythongreen}{RGB}{40,167,69}
\definecolor{pythonred}{RGB}{220,53,69}
\definecolor{pythongray}{RGB}{108,117,125}
\definecolor{codebg}{RGB}{248,249,250}
\definecolor{theorembg}{RGB}{240,248,255}
\definecolor{examplebg}{RGB}{245,255,245}
\definecolor{warningbg}{RGB}{255,250,240}
\definecolor{filozofija}{RGB}{147,112,219}
\definecolor{logika}{RGB}{255,215,0}
\definecolor{informatika}{RGB}{46,139,87}

% ===================================
% STILOVI ZA KOD (BREAKABLE VERSION)
% ===================================

\newtcblisting{pythoncode}[1][]{
    listing engine=minted,
    minted language=python,
    minted options={
        fontsize=\small,
        breaklines,
        autogobble,
        linenos,
        numbersep=5pt
    },
    colback=codebg,
    colframe=pythonblue!70,
    listing only,
    left=5mm,
    enhanced,
    breakable,
    overlay={\begin{tcbclipinterior}\fill[pythonblue!20] (frame.south west) rectangle ([xshift=5mm]frame.north west);\end{tcbclipinterior}},
    #1
}


% ===================================
% TEOREMI I OKRUŽENJA
% ===================================

\theoremstyle{definition}
\newtheorem{definicija}{Definicija}[chapter]
\newtheorem{aksiom}{Aksiom}[chapter]
\newtheorem{napomena}{Napomena}[chapter]

\theoremstyle{plain}
\newtheorem{teorem}{Teorem}[chapter]
\newtheorem{propozicija}{Propozicija}[chapter]
\newtheorem{lema}{Lema}[chapter]
\newtheorem{korolar}{Korolar}[chapter]

\theoremstyle{remark}
\newtheorem{primjer}{Primjer}[chapter]
\newtheorem{vježba}{Vježba}[chapter]

\newenvironment{definicijaokvir}
    {\begin{mdframed}[backgroundcolor=theorembg,linecolor=pythonblue,linewidth=2pt]
    \begin{definicija}}
    {\end{definicija}\end{mdframed}}

\newenvironment{primjerokvir}
    {\begin{mdframed}[backgroundcolor=examplebg,linecolor=pythongreen,linewidth=2pt]
    \begin{primjer}}
    {\end{primjer}\end{mdframed}}

\newenvironment{upozorenje}
    {\begin{mdframed}[backgroundcolor=warningbg,linecolor=pythonred,linewidth=2pt]
    \textbf{Upozorenje:} }
    {\end{mdframed}}

% ===================================
% PRILAGODBA NASLOVA
% ===================================

\titleformat{\chapter}[display]
{\normalfont\huge\bfseries\color{pythonblue}}
{\chaptertitlename\ \thechapter}{20pt}{\Huge}
\titleformat{\section}
{\normalfont\Large\bfseries\color{pythonblue!80}}
{\thesection}{1em}{}
\titleformat{\subsection}
{\normalfont\large\bfseries\color{pythonblue!60}}
{\thesubsection}{1em}{}

% ===================================
% ZAGLAVLJA I PODNOŽJA
% ===================================

\pagestyle{fancy}
\fancyhf{}
\fancyhead[LE,RO]{\thepage}
\fancyhead[LO]{\rightmark}
\fancyhead[RE]{\leftmark}
\renewcommand{\headrulewidth}{0.4pt}
\renewcommand{\footrulewidth}{0pt}
% --- FIX: Set headheight to prevent warnings ---
\setlength{\headheight}{15pt}

% ===================================
% MAKRONAREDBE
% ===================================

% --- FIX: Use \ensuremath to make command safe outside of math mode ---
\newcommand{\logsud}[1]{\ensuremath{\mathit{#1}}}
\newcommand{\konj}{\land}
\newcommand{\disj}{\lor}
\newcommand{\impl}{\rightarrow}
\newcommand{\biimp}{\leftrightarrow}
\newcommand{\svaki}{\forall}
\newcommand{\postoji}{\exists}

\newcommand{\pyinline}[1]{\texttt{\color{pythonblue}#1}}
\newcommand{\jupyter}[1]{\marginpar{\small\color{pythongreen}$\triangleright$ #1.ipynb}}


% Logic symbols
\newunicodechar{↦}{\ensuremath{\mapsto}}
\newunicodechar{□}{\ensuremath{\square}} % Requires amssymb
\newunicodechar{⊥}{\ensuremath{\bot}}
\newunicodechar{⊤}{\ensuremath{\top}}
\newunicodechar{¬}{\ensuremath{\neg}}
\newunicodechar{∧}{\ensuremath{\land}}
\newunicodechar{∨}{\ensuremath{\lor}}
\newunicodechar{→}{\ensuremath{\rightarrow}}
\newunicodechar{←}{\ensuremath{\leftarrow}}
\newunicodechar{↔}{\ensuremath{\leftrightarrow}}
\newunicodechar{⇒}{\ensuremath{\Rightarrow}}
\newunicodechar{⇐}{\ensuremath{\Leftarrow}}
\newunicodechar{⇔}{\ensuremath{\Leftrightarrow}}
\newunicodechar{∀}{\ensuremath{\forall}}
\newunicodechar{∃}{\ensuremath{\exists}}
\newunicodechar{∄}{\ensuremath{\nexists}}
\newunicodechar{⊢}{\ensuremath{\vdash}}
\newunicodechar{⊨}{\ensuremath{\models}}
\newunicodechar{⊣}{\ensuremath{\dashv}}
\newunicodechar{⊩}{\ensuremath{\Vdash}}
\newunicodechar{⊬}{\ensuremath{\nvdash}}
\newunicodechar{⊭}{\ensuremath{\nvDash}}
\newunicodechar{⊮}{\ensuremath{\nVdash}}
\newunicodechar{⊯}{\ensuremath{\nVDash}}
\newunicodechar{∈}{\ensuremath{\in}}
\newunicodechar{∉}{\ensuremath{\notin}}
\newunicodechar{⊂}{\ensuremath{\subset}}
\newunicodechar{⊃}{\ensuremath{\supset}}
\newunicodechar{⊆}{\ensuremath{\subseteq}}
\newunicodechar{⊇}{\ensuremath{\supseteq}}
\newunicodechar{⊈}{\ensuremath{\nsubseteq}}
\newunicodechar{⊉}{\ensuremath{\nsupseteq}}
\newunicodechar{∪}{\ensuremath{\cup}}
\newunicodechar{∩}{\ensuremath{\cap}}
\newunicodechar{∅}{\ensuremath{\emptyset}}
\newunicodechar{⊕}{\ensuremath{\oplus}}
\newunicodechar{⊗}{\ensuremath{\otimes}}
\newunicodechar{⊖}{\ensuremath{\ominus}}
\newunicodechar{⊙}{\ensuremath{\odot}}
\newunicodechar{✓}{\ensuremath{\checkmark}}
\newunicodechar{✗}{\ensuremath{\times}}
\newunicodechar{≈}{\ensuremath{\approx}}
\newunicodechar{Σ}{\ensuremath{\Sigma}}
\newunicodechar{δ}{\ensuremath{\delta}}
\newunicodechar{λ}{\ensuremath{\lambda}}
\newunicodechar{μ}{\ensuremath{\mu}}
\newunicodechar{π}{\ensuremath{\pi}}
\newunicodechar{ε}{\ensuremath{\varepsilon}}
\newunicodechar{Γ}{\ensuremath{\Gamma}}
\newunicodechar{≤}{\ensuremath{\leq}}
\newunicodechar{≥}{\ensuremath{\geq}}
\newunicodechar{≠}{\ensuremath{\neq}}
\newunicodechar{≡}{\ensuremath{\equiv}}
\newunicodechar{⟺}{\ensuremath{\Longleftrightarrow}}
\newunicodechar{⟹}{\ensuremath{\Longrightarrow}}
\newunicodechar{⟸}{\ensuremath{\Longleftarrow}}
\newunicodechar{₀}{\ensuremath{_0}}
\newunicodechar{₁}{\ensuremath{_1}}
\newunicodechar{₂}{\ensuremath{_2}}
\newunicodechar{₃}{\ensuremath{_3}}
\newunicodechar{₄}{\ensuremath{_4}}
\newunicodechar{₅}{\ensuremath{_5}}
\newunicodechar{₆}{\ensuremath{_6}}
\newunicodechar{₇}{\ensuremath{_7}}
\newunicodechar{₈}{\ensuremath{_8}}
\newunicodechar{₉}{\ensuremath{_9}}
\newunicodechar{ₘ}{\ensuremath{_m}}
\newunicodechar{ₖ}{\ensuremath{_k}}
\newunicodechar{ᵢ}{\ensuremath{_i}}
\newunicodechar{⁰}{\ensuremath{^0}}
\newunicodechar{¹}{\ensuremath{^1}}
\newunicodechar{²}{\ensuremath{^2}}
\newunicodechar{³}{\ensuremath{^3}}
\newunicodechar{⁴}{\ensuremath{^4}}
\newunicodechar{⁵}{\ensuremath{^5}}
\newunicodechar{⁶}{\ensuremath{^6}}
\newunicodechar{⁷}{\ensuremath{^7}}
\newunicodechar{⁸}{\ensuremath{^8}}
\newunicodechar{⁹}{\ensuremath{^9}}
\newunicodechar{ⁿ}{\ensuremath{^n}}
\newunicodechar{⁺}{\ensuremath{^+}}
\newunicodechar{∞}{\ensuremath{\infty}}
\newunicodechar{ℵ}{\ensuremath{\aleph}}
\newunicodechar{ℕ}{\ensuremath{\mathbb{N}}}
\newunicodechar{ℤ}{\ensuremath{\mathbb{Z}}}
\newunicodechar{ℚ}{\ensuremath{\mathbb{Q}}}
\newunicodechar{⚡}{\ensuremath{\star}}
\newunicodechar{⚠}{\ensuremath{\triangle}}
\newunicodechar{△}{\ensuremath{\triangle}}
\newunicodechar{β}{\ensuremath{\beta}}
\newunicodechar{ₙ}{\ensuremath{_n}}
\newunicodechar{ℝ}{\ensuremath{\mathbb{R}}}
\newunicodechar{Ω}{\ensuremath{\Omega}}
\newunicodechar{φ}{\ensuremath{\varphi}}
\newunicodechar{ψ}{\ensuremath{\psi}}

% ===================================
% DOKUMENT
% ===================================

\begin{document}

\frontmatter
\begin{titlepage}
    \thispagestyle{empty}
    \centering

    % Use \vfill to flexibly manage vertical space and prevent page overflow
    \vspace*{1.5cm} % Pushes the content down from the top margin

    %---------------------------------------------------------------
    % TITLE AND SUBTITLE
    %---------------------------------------------------------------
    {\fontsize{40}{48}\selectfont\bfseries\color{pythonblue}
    LOGIKA U KODU}

    \vspace{0.5cm}

    % A subtle decorative rule
    \rule{0.4\textwidth}{0.4pt}

    \vspace{0.8cm}

    {\Large\color{pythongray}
    Elementi logike kroz programski jezik Python}

    \vfill % Adds flexible space after the title block

    %---------------------------------------------------------------
    % VENN DIAGRAM
    % Recreates the visual from your original PDF
    %---------------------------------------------------------------
    \begin{center}
    \begin{tikzpicture}[scale=1.1, every node/.style={transform shape, text=black!70}]
        % Circles with fill and opacity to show intersections
        \begin{scope}[fill opacity=0.55, line width=1.2pt]
            \fill[logika, draw=logika!80!black] (0, 1.3) circle (2.2cm);
            \fill[filozofija, draw=filozofija!80!black] (-1.1, -0.6) circle (2.2cm);
            \fill[informatika, draw=informatika!80!black] (1.1, -0.6) circle (2.2cm);
        \end{scope}

        % Main labels for each field
        \node at (0, 2.7) {\Large\bfseries LOGIKA};
        \node[anchor=east] at (-2.3, 0.6) {\Large\bfseries FILOZOFIJA};
        \node[anchor=west] at (2.3, 0.6) {\Large\bfseries INFORMATIKA};

        % Symbols inside the main part of each circle
        \node at (0, 0.2) {\huge $\forall \exists$};
        \node at (-1.5, -1.2) {\huge $\Phi$};
        \node at (1.5, -1.2) {\huge \texttt{</>}};

%        % Labels for the intersection areas
%        \node[align=center, font=\small] at (-0.9, 0.5) {formalni\\dokazi};
%        \node[align=center, font=\small] at (0.9, 0.5) {Boolean\\logika};
%        \node[align=center, font=\small] at (0, -1.5) {etika\\AI};
        \node[font=\huge] at (0, -0.6) {λ};
    \end{tikzpicture}
    \end{center}

    \vfill % Adds flexible space between the diagram and the author info

    %---------------------------------------------------------------
    % COURSE AND AUTHOR INFORMATION
    %---------------------------------------------------------------
    {\large\scshape Priručnik za kolegij}\\[0.5cm]
    {\Large\bfseries ,,Logika i programiranje''}\\[1.5cm]
    {\huge Davor Lauc}

    \vfill % Pushes the institution details to the bottom

    %---------------------------------------------------------------
    % INSTITUTION AND YEAR
    %---------------------------------------------------------------
    \begin{center}
        {\large\scshape Filozofski fakultet}\\[0.3cm]
        {\small Sveučilište u Zagrebu}\\[1cm]
        {\huge\bfseries 2025}
    \end{center}

    \vspace*{1cm} % Adds a bit of margin at the very bottom

\end{titlepage}
\newpage
\thispagestyle{empty}
\vspace*{\fill}
\noindent
\setlength{\parindent}{0pt}
\setlength{\parskip}{\baselineskip}
\textsc{Nakladnik}: \\
Sveučilište u Zagrebu \\
Filozofski fakultet \\
FF Press \\
\vspace{0.7cm}
\textsc{Za nakladnika}:\\
\vspace{0.7cm}

\textsc{Recenzenti}: \\
Prof. dr. sc. Zdravko Dovedan Han \\
Doc. dr. sc. Ines Skelac \\
\vspace{0.7cm}

\textsc{Grafička priprema}: \\
Davor Lauc \\

\vspace{0.7cm}
\textsc{ISBN:} \\ {\bfseries 978-XXX-XXX-XXX-X}
\textsc{DOI:} \\ {\bfseries 10.17234/XXXXXXXXXX}

\vspace{0.7cm}
\textit{Prvo izdanje, Zagreb, 2025}

\vspace{0.7cm}
Djelo je objavljeno pod uvjetima Creative Commons Autorstvo-Nekomercijalno-Dijeli pod istim uvjetima 4.0 Međunarodne javne licence (CC-BY-NC-SA) koja dopušta korištenje, dijeljenje i umnažanje djela, ali samo u nekomercijalne svrhe i uz uvjet da se ispravno citira djelo i autora, te uputi na izvor.
Prerađena djela moraju se objaviti pod istim uvjetima.
\vspace{0.5cm}

\textsc{Napomena}: \\
Svi Python primjeri koda u ovom udžbeniku dostupni su kao interaktivne Jupyter bilježnice na adresi: \\

\texttt{https://github.com/ffzg-logika/logika-u-kodu}
\vspace*{\fill}

\tableofcontents

\chapter*{Predgovor}
\addcontentsline{toc}{chapter}{Predgovor}

Ovaj udžbenik nastao je iz više godina predavanja kolegija „Logika i programiranje" na Filozofskom fakultetu
Sveučilišta u Zagrebu.
Kroz godine rada sa studentima filozofije koji su prvi put pisali kod,
kao i sa studentima informatike koji su otkrivali filozofske temelje svojih programa,
kristalizirao se  pristup koji spaja apstraktno i konkretno, teoriju i praksu.

\section*{Geneza projekta}

Početna ideja bila je jednostavna: učiniti formalnu logiku opipljivom i zanimljivijom, omogućiti transfer vještina između formalne logike i programskih vještina,
te pružiti studentima filozofije dodatne ,,zapošljive'' vještine.
Umjesto da studenti samo vježbaju izvode prirodne dedukcije, zašto ih ne bi implementirali i tako bolje razumjeli?
Umjesto da crtaju istinosne tablice na papiru, zašto ih ne bi generirali programski?
Ono što je počelo kao eksperiment, pretvorilo se u potpuno novi način podućavanja logike.

\section*{Struktura udžbenika}

Knjiga je organizirana u dva komplementarna dijela:

\textbf{Prvi dio – Svjetovi deduktivne logike} pokriva klasične sustave gdje zaključci nužno slijede iz premisa. Počinjemo s Wittgensteinovom slikom svijeta kao skupa činjenica, prelazimo na Gentzenove sustave prirodne dedukcije, istražujemo Tarskijevu semantiku, te završavamo s računskim aspektima kroz Turingove strojeve.

\textbf{Drugi dio – Svjetovi induktivnih logika} istražuje sustave gdje zaključci imaju samo određeni stupanj vjerojatnosti. Od Pascalove oklade preko Bayesova teorema do suvremenih pristupa u strojnom učenju.

Svako poglavlje slijedi konzistentnu strukturu: motivacija, formalizacija, implementacija, eksploracija.
Ova četverodijelna struktura omogućava različite razine angažmana – od površnog upoznavanja do dubinskog istraživanja.
Neki djelovi se djelomično preklapaju radi održavanja cjeline studenskih izlaganja.

\subsection*{Pedagoški pristup}

Kroz godine predavanja, razvio sam nekoliko ključnih principa:

\textbf{Greške su pedagoški momenti.} Kada studentov kod ne radi, to nije neuspjeh već prilika za razumijevanje zašto logička pravila funkcioniraju kako funkcioniraju.

\textbf{Apstrakcija kroz konkretno.} Svaki apstraktni koncept ima konkretnu implementaciju koju možete pokrenuti,
modificirati i igrati se s njom.

\textbf{Spiralno učenje.} Isti koncepti vraćaju se na različitim razinama složenosti. I
mplikacija se prvo pojavljuje kao Python \texttt{if-then}, zatim kao materijalna kondicionala,
pa kao pravilo u prirodnoj dedukciji, i konačno kao tip funkcije, i metalogički odnos.

\section*{Za koga je ova knjiga?}

Primarno, za studente filozofije koji žele razumjeti formalnu logiku kroz praktičnu primjenu.
Ali kroz godine, publika se proširila:

- Studenti računarstva pronalaze filozofske temelje svoje discipline
- Nastavnici srednji škola koriste materijale za modernizaciju nastave
- Istraživači u AI-ju pronalaze korisne implementacije klasičnih sustava
- Autodidakti koji uživaju u samostalnom istraživanju

Ne pretpostavljamo prethodno znanje programiranja – dodatak A pruža sve potrebne osnove Pythona.
Također ne pretpostavljamo formalno obrazovanje iz logike – gradimo od temelja.

\section*{Kako koristiti materijale}

Sav kod dostupan je na \texttt{https://github.com/dlauc/logikaukodu}. Bilježnice možete:
- Pokretati lokalno s Jupyter instalacijom
- Koristiti kroz Google Colab bez instalacije
- Modificirati za vlastite potrebe
- Integrirati u vlastite kolegije

Licenca \em{Creative Commons} omogućava slobodno dijeljenje i adaptaciju uz navođenje izvora.

\section*{Napomena o budućnosti}

Ovaj udžbenik nije završen proizvod već živi dokument. Svaki semestar donosi nove uvide, nove načine objašnjavanja,
nove veze između koncepata.
Pozivam čitatelje da se uključe na djeljenji kod – prijavite greške, predložite poboljšanja, podijelite svoje implementacije.
Logika i programiranje nisu samo akademske discipline – oni su načini mišljenja koji oblikuju našu digitalnu stvarnost.
Nadam se da će ova knjiga pomoći novim generacijama da ovladaju oba.

\vspace{1cm}
\begin{flushright}
\textit{Davor Lauc}\\
\textit{Zagreb, rujan 2025}
\end{flushright}


\mainmatter

\chapter{Uvod: Što je logika i zašto kod?}

\epigraph{Ne priliči izvrsnim ljudima da kao robovi gube sate na računanje, koji se posao može sa punim povjerenjem prepustiti bilo kome drugome uporabom strojeva.}{Gottfried Wilhelm Leibniz\footnotemark}
\footnotetext{\emph{Machina Arithmetica in qua non Aditio tantum Subtractio 1685}, slobodni prijevod s engleskog prijevoda}

\section*{Logika kao znanost o valjanom zaključivanju}

Logika je znanost koja proučava oblike valjanog zaključivanja i relacije logičkog slijeda. Kada kažemo da je zaključak valjan, mislimo da konkluzija nužno slijedi iz premisa. Ova nužnost nije slučajna – ona proizlazi iz same strukture našeg mišljenja.

Razmotrimo jednostavan primjer. Ako znamo da ``Svi programi imaju greške'' i da je ``Python program'', možemo zaključiti da ``Python ima greške''. Ovo zaključivanje ne ovisi o tome što mislimo o programima ili Pythonu -- ono je valjano zbog svoje logičke forme.

U suvremenoj simboličkoj logici, ovu formu možemo precizno zapisati. Logika sudova bavi se osnovnim veznicima poput konjunkcije ($\wedge$), disjunkcije ($\vee$) i implikacije ($\rightarrow$). Logika predikata ide korak dalje i omogućava nam kvantificiranje -- govorenje o ``svim'' objektima ($\forall$) ili ``nekim'' objektima ($\exists$).

\section*{Python kao jezik za istraživanje logičkih principa}

Zašto baš Python? Ovaj programski jezik nudi idealnu ravnotežu između jednostavnosti i moći. Njegova sintaksa je bliska prirodnom jeziku, što omogućava lakše razumijevanje logičkih koncepata. Istovremeno, Python je dovoljno moćan da implementira složene logičke sustave.

Kada u Pythonu pišemo:
\begin{verbatim}
if pada_kiša and nemam_kišobran:
    postat_ću_mokar = True
\end{verbatim}

Mi zapravo koristimo logičku konjunkciju. Uvjet \texttt{pada\_kiša and nemam\_kišobran} je istinit samo ako su oba dijela istinita -- upravo kao logički veznik $\wedge$.

Python nam omogućava da logičke koncepte učinimo opipljivima. Možemo implementirati istinosne tablice, provjeriti valjanost zaključaka i vizualizirati logičke strukture. Kroz kod, apstraktni logički principi postaju konkretni i provjerljivi.

\section*{Filozofska pozadina i praktična primjena}

Veza između logike i računarstva nije slučajna. Oba područja bave se preciznim opisivanjem i manipuliranjem informacija. Gottlob Frege, utemeljitelj moderne logike, želio je stvoriti ``pojmovno pismo'' (\textit{Begriffsschrift}) -- formalni jezik za izražavanje čistog mišljenja. Danas, programski jezici ostvaruju upravo tu viziju.

Ludwig Wittgenstein u svom \textit{Tractatusu} tvrdi da logika pokazuje strukturu stvarnosti. Kada pišemo program, mi zapravo opisujemo moguće svjetove -- kombinacije stanja koje naš sustav može poprimiti. Svaka \texttt{if-else} struktura definira grananje između mogućih svjetova.

Praktične primjene ove veze su svugdje oko nas. Sustavi za automatsko zaključivanje pomažu u medicinskoj dijagnostici. Formalna verifikacija osigurava ispravnost kritičnih sustava poput autopilota u zrakoplovima. Baze podataka koriste logiku predikata za učinkovito pretraživanje informacija.


\section*{Curry-Howard izomorfizam: Most između svjetova}

Jedan od najfascinantnijih rezultata 20. stoljeća je otkriće da postoji duboka matematička veza između logičkih dokaza i računalnih programa. Ovu vezu, poznatu kao Curry-Howard izomorfizam, neovisno su otkrili Haskell Curry i William Alvin Howard.

Izomorfizam pokazuje sljedeće ekvivalencije:
\begin{align}
\text{Logičke formule} &\leftrightarrow \text{Tipovi u programiranju}\\
\text{Dokazi} &\leftrightarrow \text{Programi}\\
\text{Normalizacija dokaza} &\leftrightarrow \text{Izvršavanje programa}
\end{align}

Što to znači? Svaki put kada pišete funkciju tipa \texttt{A -> B}, vi zapravo tvrdite da možete dokazati B iz A. Kada pozovete tu funkciju s argumentom tipa A, vi primjenjujete modus ponens!

Razmotrimo konkretan primjer. Logička formula $P \rightarrow (Q \rightarrow P)$ kaže: ``Ako je P istinito, onda bez obzira na Q, P je istinito.'' U Pythonu, ova formula odgovara funkciji:
\begin{verbatim}
def konstanta(p):
    def ignoriraj(q):
        return p
    return ignoriraj
\end{verbatim}

Funkcija prima vrijednost \texttt{p} i vraća funkciju koja ignorira svoj argument i uvijek vraća \texttt{p}. To je dokaz logičke formule pretočen u program!

\section*{Načini mišljenja: Logičko i računalno}

Logičko i računalno mišljenje dijele ključne vještine:

\textbf{Apstrakcija:} I logičari i programeri moraju identificirati bitne strukture i zanemariti nevažne detalje. Kada kažemo ``Svi ljudi su smrtni'', apstrahiramo od individualnih razlika. Kada pišemo funkciju, apstrahiramo od konkretnih vrijednosti.

\textbf{Dekompozicija:} Složene probleme rastavljamo na jednostavnije. U logici, složene formule analiziramo kroz njihove komponente. U programiranju, velike programe gradimo od manjih funkcija.

\textbf{Preciznost:} I logika i programiranje zahtijevaju precizno izražavanje. Dvosmislenost koja je prihvatljiva u svakodnevnom govoru može dovesti do pogrešnih zaključaka ili bugova u kodu.

\textbf{Rekurzija:} Koncept koji je ključan u oba područja. U logici, možemo definirati beskonačne strukture pomoću konačnih pravila. U programiranju, rekurzivne funkcije omogućavaju elegantna rješenja složenih problema.

\section*{Od temelja prema naprijed}

Ovaj udžbenik vodi vas kroz postupno otkrivanje veze između logike i programiranja. Počinjemo s osnovama -- što su sudovi i kako ih kombinirati. Kroz Python implementacije, ove apstraktne koncepte činimo konkretnima i provjerljivima.

Kako napredujete, otkrit ćete da logika nije samo teoretska disciplina. Ona je alat koji omogućava:
\begin{itemize}
\item Jasnije razmišljanje o problemima
\item Preciznije izražavanje ideja
\item Sustavnije pristupanje složenim zadacima
\item Pouzdanije zaključivanje
\end{itemize}

Svaki koncept koji naučite ima direktnu primjenu u programiranju. Razumijevanje logike sudova pomaže u pisanju uvjetnih izraza. Logika predikata ključna je za rad s bazama podataka. Prirodna dedukcija povezana je s tipskim sustavima modernih programskih jezika.

\section*{Poziv na istraživanje}

Logika i programiranje nisu samo alati -- oni su načini razumijevanja svijeta. Kroz ovaj udžbenik, pozivamo vas da istražite kako formalni sustavi mogu opisati i stvarnost i mišljenje o stvarnosti.

Svaka Jupyter bilježnica je laboratorij u kojem možete eksperimentirati. Mijenjajte kod, testirajte granične slučajeve, pokušajte slomiti sustave. Kroz ovo aktivno istraživanje, razvit ćete intuiciju koja je jednako važna kao i formalno znanje.

Zapamtite: u logici, kao i u programiranju, greške su prilike za učenje. Kada program ne radi kako očekujete, ili kada zaključak nije valjan kako ste mislili, vi otkrivate suptilnosti logičkih principa. Ova otkrića su najvrjedniji dio učenja.

Krenimo zajedno na ovo putovanje kroz logiku i kod, gdje svaki novi koncept otvara vrata dubljeg razumijevanja i načina na koji mislimo, zaključujemo i stvaramo.

\section*{Tehničke napomene}

Jupyter bilježnice omogućavaju nam jedinstveni pristup učenju logike. U istom dokumentu možemo kombinirati:
\begin{itemize}
\item LaTeX formule za precizno zapisivanje logičkih izraza
\item Python kod za implementaciju i testiranje koncepata
\item Vizualizacije koje čine apstraktne ideje jasnima
\item Objašnjenja koja povezuju teoriju s praksom
\end{itemize}

Ovaj pristup omogućava interaktivno istraživanje.
Umjesto pasivnog čitanja o modus ponensu, možete ga implementirati, testirati na različitim primjerima i vidjeti kako radi.
Greške postaju prilike za učenje -- kada program ne radi kako očekujete, otkrivate nijanse logičkih pravila.

Sve bilježnice iz kojih je nastala ova knjiga nalaze se na \url{https://github.com/dlauc/logikaukodu}, te se mogu pokrenuti lokalni ili na sustavima poput Google Colabb-a.


\part{SVJETOVI DEDUKTIVNE LOGIKE}
\input{wittgenstein}
\input{gentzen}
\input{tarski}
\input{turing}
\input{cantor}

\part{SVJETOVI INDUKTIVNIH LOGIKA}
\input{pascal}
\input{bayes}
\input{goodman}

\part{DODACI}

\appendix
 
\chapter{Uvod u Python za studente filozofije i ostalih ne-tehničkih grupa)}
\label{chap:Uvod u Pzthon}

\section{Zašto bi se filozof zanimao za programiranje?}

Na prvi pogled, svijet filozofije i svijet programiranja mogu se činiti kao dva potpuno odvojena svemira. Jedan se bavi vječnim pitanjima o smislu, postojanju i vrijednostima, dok se drugi bavi preciznim uputama za strojeve. Međutim, ispod površine, ova dva svijeta dijele duboke i iznenađujuće veze. Logika, temeljni alat filozofske analize, ujedno je i srce svakog računalnog programa. Način na koji strukturiramo argumente, definiramo pojmove i izvodimo zaključke u filozofiji ima svoj odraz u načinu na koji pišemo kod.

Učenje programskog jezika \pyinline{Python}, stoga, za studenta filozofije nije samo stjecanje tehničke vještine, već i prilika za istraživanje poznatih koncepata iz nove perspektive. Kroz \pyinline{Python}, apstraktni pojmovi poput varijabli, uvjeta i petlji postaju konkretni alati s kojima možete raditi, eksperimentirati i stvarati.

Ovo poglavlje je osmišljeno kao blagi uvod u \pyinline{Python}, posebno prilagođen studentima humanističkih i društvenih znanosti. Nećemo se baviti složenim matematičkim problemima niti dubokim tehničkim detaljima. Umjesto toga, fokusirat ćemo se na osnove jezika, koristeći primjere koji su vam bliski: analizu teksta, rad s riječima i rečenicama, te istraživanje ideja kroz kod. Koristit ćemo se \textbf{Jupyter bilježnicama}, interaktivnim okruženjem koje omogućuje pisanje koda, teksta i vizualizacija na jednom mjestu, čineći učenje intuitivnim i zabavnim.

Dok budete prolazili kroz ovo poglavlje, potičem vas da ne gledate na kod samo kao na niz naredbi, već kao na novi način izražavanja i strukturiranja misli. Možda ćete otkriti da vam učenje programiranja može pomoći da postanete precizniji u svom filozofskom promišljanju, jasniji u svom izražavanju i kreativniji u svom pristupu problemima. Dobrodošli u svijet \pyinline{Python}a!

\section{Osnovni pojmovi: Varijable, tipovi podataka i izrazi}
\label{sec:osnovnipojmovi}

\subsection{Varijable: Imenovanje ideja}

U filozofiji, često koristimo simbole ili nazive kako bismo predstavili složene ideje. Na primjer, u logici, slovo \logsud{P} može predstavljati propoziciju "Svi ljudi su smrtni". Na sličan način, u \pyinline{Python}u koristimo \textbf{varijable} kao imenovane spremnike za pohranu podataka.

\begin{definicijaokvir}
    \textbf{Varijabla} je imenovani prostor u memoriji koji služi za pohranu vrijednosti. Ime varijable (identifikator) koristimo kako bismo pristupili pohranjenoj vrijednosti.
\end{definicijaokvir}

Varijablu možete zamisliti kao oznaku koju pridružujete nekoj vrijednosti. Operator dodjele, znak jednakosti (\pyinline{=}), koristi se za dodjeljivanje vrijednosti varijabli.

\begin{primjerokvir}
    Dodjeljivanje vrijednosti varijablama.
    \begin{pythoncode}
pozdrav = "Zdravo, svijete!"
godinarodenjakanta = 1724
pipriblizno = 3.14159
    \end{pythoncode}
    U ovom primjeru, \pyinline{pozdrav}, \pyinline{godinarodenjakanta} i \pyinline{pipriblizno} su nazivi varijabli. Jednom kada definiramo varijablu, možemo je koristiti u daljnjem kodu, na primjer za ispis njezine vrijednosti pomoću ugrađene funkcije \pyinline{print()}.
    \begin{pythoncode}
print(pozdrav)
print(godinarodenjakanta)
    \end{pythoncode}
    \begin{codeoutput}
Zdravo, svijete!
1724
    \end{codeoutput}
\end{primjerokvir}


\subsection{Tipovi podataka: Različite vrste informacija}

U filozofiji, razlikujemo različite vrste pojmova: konkretne, apstraktne, pojedinačne, opće. Slično tome, u \pyinline{Python}u, svaka vrijednost pripada određenom \textbf{tipu podataka}. Osnovni tipovi podataka koje ćemo za početak koristiti su:
\begin{itemize}[leftmargin=*]
    \item \textbf{String (\pyinline{str}):} Niz znakova, odnosno tekstualni podaci. Stringovi se uvijek pišu unutar navodnika (jednostrukih \pyinline{''} ili dvostrukih \pyinline{""}). Primjeri: \pyinline{"Sokrat"}, \pyinline{'Platonova Država'}.
    \item \textbf{Integer (\pyinline{int}):} Cijeli brojevi, bez decimalnog dijela. Primjeri: \pyinline{42}, \pyinline{-399}, \pyinline{2025}.
    \item \textbf{Float (\pyinline{float}):} Brojevi s pomičnim zarezom (decimalni brojevi). Primjeri: \pyinline{3.14}, \pyinline{9.81}, \pyinline{-0.5}.
    \item \textbf{Boolean (\pyinline{bool}):} Logička ili istinitosna vrijednost. Može imati samo dvije vrijednosti: \pyinline{True} (istina) ili \pyinline{False} (laž).
\end{itemize}

\pyinline{Python} je dinamički tipiziran jezik, što znači da ne moramo unaprijed deklarirati tip varijable. Interpretator automatski prepoznaje tip podatka kada dodijelimo vrijednost. Tip varijable možemo provjeriti pomoću ugrađene funkcije \pyinline{type()}.

\begin{primjerokvir}
    Provjera tipova podataka.
    \begin{pythoncode}
filozof = "Aristotel"
godinarodenja = -384
visinaumetrima = 1.7
jeliziv = False

print(type(filozof))
print(type(godinarodenja))
print(type(visinaumetrima))
print(type(jeliziv))
    \end{pythoncode}
    \begin{codeoutput}
<class 'str'>
<class 'int'>
<class 'float'>
<class 'bool'>
    \end{codeoutput}
\end{primjerokvir}

\subsection{Izrazi: Kombiniranje vrijednosti}

U logici, kombiniramo propozicije pomoću veznika ($\konj, \disj, \impl$) kako bismo stvorili složenije izraze. U \pyinline{Python}u, \textbf{izrazi} su kombinacije vrijednosti, varijabli i operatora koje se izračunavaju (evaluiraju) kako bi proizvele novu vrijednost.

\begin{itemize}[leftmargin=*]
    \item \textbf{Aritmetički izrazi:} Koriste standardne matematičke operatore (\pyinline{+}, \pyinline{-}, \pyinline{*}, \pyinline{/}).
    \begin{pythoncode}
a = 10
b = 5
zbroj = a + b
print(zbroj)
    \end{pythoncode}
    \begin{codeoutput}
15
    \end{codeoutput}

    \item \textbf{String izrazi:} Operator \pyinline{+} se može koristiti za spajanje (\textit{konkatenaciju}) stringova.
    \begin{pythoncode}
ime = "Immanuel"
prezime = "Kant"
punoime = ime + " " + prezime
print(punoime)
    \end{pythoncode}
    \begin{codeoutput}
Immanuel Kant
    \end{codeoutput}
\end{itemize}

\section{Strukture podataka: Organiziranje misli}
\label{sec:strukturepodataka}

Dok osnovni tipovi podataka predstavljaju pojedinačne vrijednosti, \textbf{strukture podataka} služe za organiziranje i pohranu više vrijednosti u jednoj varijabli.

\subsection{Liste: Uređeni nizovi argumenata}

U filozofskim tekstovima, često nailazimo na nabrajanja ili nizove ideja, poput Aristotelovih četiriju uzroka. U \pyinline{Python}u, \textbf{liste} su strukture podataka koje nam omogućuju pohranu uređenog niza elemenata. Elementi liste se navode unutar uglatih zagrada \pyinline{[]}, odvojeni zarezima.

\begin{definicijaokvir}
    \textbf{Lista} (\pyinline{list}) je promjenjiva, uređena kolekcija elemenata. "Uređena" znači da elementi zadržavaju redoslijed kojim su dodani. "Promjenjiva" znači da možemo dodavati, uklanjati ili mijenjati elemente nakon što je lista stvorena.
\end{definicijaokvir}

\begin{primjerokvir}
    Kreiranje i pristupanje elementima liste.
    \begin{pythoncode}
aristoteloviuzroci = ["materijalni", "formalni", "djelatni", "svršni"] # Popiy Aristotelovih uzroka

# Pristupanje elementima pomoću indeksa
# Indeksiranje počinje od 0!
prviuzrok = aristoteloviuzroci[0]
treciuzrok = aristoteloviuzroci[2]

print("Prvi uzrok je:", prviuzrok)
print("Treći uzrok je:", treciuzrok)
    \end{pythoncode}

    \begin{codeoutput}
Prvi uzrok je: materijalni
Treći uzrok je: djelatni
    \end{codeoutput}

    Liste su promjenjive. Možemo im dodavati nove elemente metodom \pyinline{append()} ili uklanjati postojeće metodom \pyinline{remove()}.

    \begin{pythoncode}
aristoteloviuzroci.append("imaginarni") # Dodavanje petog, "imaginarnog" uzroka
print(aristoteloviuzroci)

# Uklanjanje "imaginarnog" uzroka
aristoteloviuzroci.remove("imaginarni")
print(aristoteloviuzroci)
    \end{pythoncode}
    \begin{codeoutput}
['materijalni', 'formalni', 'djelatni', 'svršni', 'imaginarni']
['materijalni', 'formalni', 'djelatni', 'svršni']
    \end{codeoutput}
\end{primjerokvir}

\subsection{Rječnici: Asocijativni parovi pojmova i definicija}

U filozofiji, često definiramo pojmove tako da im pridružujemo njihove definicije. U \pyinline{Python}u, \textbf{rječnici} (\pyinline{dict}) omogućuju pohranu podataka u obliku parova \textbf{ključ-vrijednost}. Ključevi su jedinstveni i koriste se za pristup pripadajućim vrijednostima.

\begin{definicijaokvir}
    \textbf{Rječnik} (\pyinline{dict}) je promjenjiva, neuređena kolekcija parova ključ-vrijednost. Svaki ključ mora biti jedinstven unutar rječnika.
\end{definicijaokvir}

Rječnici se definiraju unutar vitičastih zagrada \pyinline{{}}, a parovi ključ-vrijednost odvojeni su dvotočkom.

\begin{primjerokvir}
    Kreiranje i korištenje rječnika.
    \begin{pythoncode}
filozofskirjecnik = {
    "epistemologija": "grana filozofije koja se bavi znanjem",
    "metafizika": "grana filozofije koja se bavi prvim uzrocima i principima bića",
    "etika": "grana filozofije koja se bavi moralom"
}

# Pristupanje vrijednosti pomoću ključa
definicijaetike = filozofskirjecnik["etika"]
print(definicijaetike)

# Dodavanje novog para
filozofskirjecnik["logika"] = "znanost o metodama i principima ispravnog zaključivanja"
print(filozofskirjecnik["logika"])
    \end{pythoncode}
    \begin{codeoutput}
grana filozofije koja se bavi moralom
znanost o metodama i principima ispravnog zaključivanja
    \end{codeoutput}
\end{primjerokvir}

\section{Kontrola toka: Usmjeravanje argumentacije}
\label{sec:kontrolatoka}

Programi se ne izvršavaju uvijek linearno, od prve do zadnje naredbe. \textbf{Kontrola toka} odnosi se na naredbe koje nam omogućuju da usmjeravamo tijek izvršavanja programa, donosimo odluke i ponavljamo operacije.

\subsection{Uvjetno izvršavanje: \pyinline{if}, \pyinline{elif}, \pyinline{else}}

U filozofskoj argumentaciji, često koristimo uvjetne rečenice oblika "Ako $\logsud{P}$, onda $\logsud{Q}$". U \pyinline{Python}u, \textbf{uvjetne naredbe} nam omogućuju da izvršimo određeni dio koda samo ako je zadovoljen neki uvjet. Uvjet je izraz koji se evaluira kao \pyinline{True} ili \pyinline{False}.

\begin{primjerokvir}
    Korištenje \pyinline{if-else} strukture.
    \begin{pythoncode}
tvrdnja = "Sokrat je smrtan"

if "Sokrat" in tvrdnja:
    print("Tvrdnja se odnosi na Sokrata.")
else:
    print("Tvrdnja se ne odnosi na Sokrata.")
    \end{pythoncode}
    \begin{codeoutput}
Tvrdnja se odnosi na Sokrata.
    \end{codeoutput}

    Možemo koristiti i \pyinline{elif} (skraćeno od \textit{else if}) za provjeru više uzastopnih uvjeta.

    \begin{pythoncode}
godina = 1804

if godina < 476:
    print("Antička filozofija")
elif 476 <= godina < 1500:
    print("Srednjovjekovna filozofija")
else:
    print("Moderna i suvremena filozofija")
    \end{codeoutput}
    \begin{codeoutput}
Moderna i suvremena filozofija
    \end{codeoutput}
\end{primjerokvir}

\subsection{Ponavljanje: \pyinline{for} petlja}

Često je potrebno ponoviti istu radnju više puta. Na primjer, analizirati svaku riječ u rečenici. U \pyinline{Python}u, \textbf{\pyinline{for} petlja} nam omogućuje da iteriramo (prolazimo) kroz elemente sekvence (poput liste ili stringa) i za svaki element izvršimo određeni blok koda.

\begin{primjerokvir}
    Iteriranje kroz listu pomoću \pyinline{for} petlje.
    \begin{pythoncode}
stoickevrline = ["mudrost", "pravednost", "hrabrost", "umjerenost"]

print("Prema stoicima, temeljne vrline su:")
for vrlina in stoickevrline:
    print("- " + vrlina)
    \end{pythoncode}
    \begin{codeoutput}
Prema stoicima, temeljne vrline su:
- mudrost
- pravednost
- hrabrost
- umjerenost
    \end{codeoutput}

    U ovom primjeru, varijabla \pyinline{vrlina} se naziva \textit{varijabla petlje}. U svakom prolasku (iteraciji) kroz petlju, ona poprima vrijednost sljedećeg elementa iz liste \pyinline{stoickevrline}.
\end{primjerokvir}

\section{Funkcije: Modularizacija i ponovna upotreba misli}
\label{sec:funkcije}

U filozofiji, kompleksne ideje često razlažemo na manje, razumljivije dijelove. U \pyinline{Python}u, \textbf{funkcije} nam omogućuju da grupiramo niz naredbi u logičku cjelinu koju možemo pozvati više puta. Time se izbjegava ponavljanje koda i programi postaju organiziraniji i lakši za čitanje.

\begin{definicijaokvir}
    \textbf{Funkcija} je imenovani blok koda koji izvršava određeni zadatak. Može primiti ulazne podatke (argumente) i vratiti izlaznu vrijednost.
\end{definicijaokvir}

Funkcije definiramo pomoću ključne riječi \pyinline{def}.

\begin{primjerokvir}
    Definiranje i pozivanje jednostavne funkcije.
    \begin{pythoncode}
def pozdravifilozofa(ime):
    """
    Ova funkcija ispisuje pozdrav filozofu čije je ime
    prolijeđeno kao argument.
    """
    print("Pozdrav, " + ime + "!")

# Pozivanje funkcije
pozdravifilozofa("Platon")
pozdravifilozofa("Nietzsche")
    \end{pythoncode}
    \begin{codeoutput}
Pozdrav, Platon!
Pozdrav, Nietzsche!
    \end{codeoutput}
    Tekst unutar trostrukih navodnika odmah nakon definicije funkcije naziva se \textit{docstring} i služi kao dokumentacija funkcije.
\end{primjerokvir}

Funkcije mogu vraćati vrijednost pomoću naredbe \pyinline{return}. Vraćena vrijednost se tada može pohraniti u varijablu ili koristiti u daljnjim izrazima.

\begin{primjerokvir}
    Funkcija koja vraća vrijednost.
    \begin{pythoncode}
def sastaviime(ime, prezime):
    """Sastavlja puno ime iz dva dijela."""
    return ime + " " + prezime

punoimefilozofa = sastaviime("Simone", "de Beauvoir")
print(punoimefilozofa)
    \end{pythoncode}
    \begin{codeoutput}
Simone de Beauvoir
    \end{codeoutput}
\end{primjerokvir}

\section{Primjer iz prakse: Analiza filozofskog teksta}
\label{sec:primjerprakse}

Sada ćemo primijeniti sve što smo naučili na konkretnom primjeru: analizi kratkog filozofskog teksta. Cilj nam je prebrojati koliko se puta svaka riječ pojavljuje u poznatoj Descartesovoj izreci.

\begin{primjerokvir}
    Brojanje riječi u tekstu.
    \begin{pythoncode}
tekst = "Mislim, dakle jesam. Jesam, dakle postojim." # Korak 1: Definiramo tekst za analizu
print("Originalni tekst:", tekst)

# Korak 2: Priprema teksta
# Pretvaramo sva slova u mala slova kako 'Mislim' i 'mislim' ne bi bili različite riječi
tekstmali = tekst.lower()
# Uklanjamo interpunkcijske znakove
tekstbeztocke = tekstmali.replace('.', '')
tekstcisti = tekstbeztocke.replace(',', '')
print("Očišćeni tekst:", tekstcisti)

# Korak 3: Tokenizacija - razdvajanje teksta u listu riječi
rijeci = tekstcisti.split()
print("Lista riječi:", rijeci)

# Korak 4: Brojanje riječi pomoću rječnika
brojacrijeci = {}
for rijec in rijeci:
    if rijec in brojacrijeci:
        # Ako riječ već postoji u rječniku, povećaj brojač za 1
        brojacrijeci[rijec] = brojacrijeci[rijec] + 1
    else:
        # Ako je ovo prvo pojavljivanje riječi, dodaj je u rječnik s vrijednošću 1
        brojacrijeci[rijec] = 1

# Korak 5: Ispis rezultata
print("\nFrekvencija riječi:")
for rijec, broj in brojacrijeci.items():
    print(f"'{rijec}': {broj}")
    \end{pythoncode}

    \begin{codeoutput}
Originalni tekst: Mislim, dakle jesam. Jesam, dakle postojim.
Očišćeni tekst: mislim dakle jesam jesam dakle postojim
Lista riječi: ['mislim', 'dakle', 'jesam', 'jesam', 'dakle', 'postojim']

Frekvencija riječi:
'mislim': 1
'dakle': 2
'jesam': 2
'postojim': 1
    \end{codeoutput}
    Ovaj primjer integrira varijable, stringove i njihove metode (\pyinline{.lower()}, \pyinline{.replace()}, \pyinline{.split()}), liste, rječnike, \pyinline{for} petlju i \pyinline{if-else} uvjetnu logiku kako bi se riješio konkretan problem iz domene analize teksta.
\end{primjerokvir}

\section{Zaključak: Sljedeći koraci}
\label{sec:zakljucak}

Ovo poglavlje pružilo je kratak pregled osnovnih elemenata programskog jezika \pyinline{Python}. Vidjeli smo kako varijable i tipovi podataka predstavljaju osnovne gradivne blokove, kako strukture podataka poput lista i rječnika organiziraju informacije, kako kontrola toka usmjerava izvršavanje programa i kako funkcije omogućuju modularnost i ponovnu upotrebu koda.

Kao studenti filozofije, sada imate temelj za daljnje istraživanje. Sljedeći koraci mogli bi uključivati:
\begin{itemize}
    \item \textbf{Rad s tekstom:} \pyinline{Python} je izuzetno moćan za analizu teksta. Možete istražiti kako brojati riječi, analizirati sentiment, tražiti određene pojmove u velikim tekstualnim korpusima (npr. djelima pojedinih filozofa) i još mnogo toga. Biblioteke poput \pyinline{NLTK} (Natural Language Toolkit) i \pyinline{spaCy} otvaraju vrata svijeta \textit{računalne lingvistike}.
    \item \textbf{Vizualizacija podataka:} Pomoću biblioteka kao što su \pyinline{Matplotlib} i \pyinline{Seaborn}, možete vizualizirati odnose između pojmova, učestalost riječi ili druge uvide koje dobijete analizom teksta, pretvarajući apstraktne podatke u jasne grafove.
    \item \textbf{Web scraping:} Možete naučiti kako automatski prikupljati tekstualne podatke s web stranica, na primjer, s filozofskih enciklopedija ili online arhiva.
\end{itemize}

Najvažnije je da se ne bojite eksperimentirati. Jupyter bilježnice su idealno okruženje za to. Pokušajte mijenjati primjere, postavljati si vlastite male probleme i tražiti rješenja. Programiranje, kao i filozofija, je vještina koja se razvija kroz praksu, znatiželju i upornost. Sretno s kodiranjem!

\chapter*{Vježbe za poglavlje \ref{chap:uvod}}
\addcontentsline{toc}{chapter}{Vježbe za poglavlje \thechapter}

\begin{vježba}
    Kreirajte rječnik koji sadrži pet vaših omiljenih filozofa kao ključeve, a njihove glavne filozofske ideje ili djela kao vrijednosti. Zatim, koristeći \pyinline{for} petlju, ispišite svakog filozofa i njegovu ideju u formatu: \texttt{Ime Filozofa: Glavna ideja}.
\end{vježba}

\begin{vježba}
    Napišite funkciju pod nazivom \pyinline{brojrijeci} koja prima jedan argument (string) i vraća broj riječi u tom stringu. (Savjet: metoda \pyinline{split()} bi mogla biti korisna). Testirajte funkciju s nekoliko rečenica.
\end{vježba}

\begin{vježba}
    Napišite program koji provjerava pripada li godina određenom filozofskom razdoblju.
    \begin{enumerate}
        \item Definirajte listu koja sadrži nekoliko filozofa egzistencijalizma, npr. \pyinline{egzistencijalisti = ["Sartre", "Camus", "Kierkegaard"]}.
        \item Pitajte korisnika da unese ime filozofa pomoću funkcije \pyinline{input()}.
        \item Koristeći \pyinline{if} naredbu i operator \pyinline{in}, provjerite nalazi li se uneseno ime u vašoj listi.
        \item Ispišite odgovarajuću poruku, npr. \texttt{Sartre je egzistencijalist.} ili \texttt{Platon nije egzistencijalist.}.
    \end{enumerate}
\end{vježba}

\backmatter
% \bibliographystyle{plain}
% \bibliography{literatura}

\end{document}