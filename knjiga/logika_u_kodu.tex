
\documentclass[11pt,a4paper,twoside,openright]{book}

% ===================================
% PAKETI
% ===================================

% Osnovni paketi za hrvatski jezik
\usepackage{fontspec}        % Omogućuje korištenje OpenType fontova u XeTeX/LuaTeX
\usepackage[croatian]{babel}
\usepackage{lmodern}         % I dalje je dobro imati za matematičke fontove

% Postavljanje zadanih fontova na Latin Modern (moderna verzija)
\setmainfont{Latin Modern Roman}
\setsansfont{Latin Modern Sans}
\setmonofont{DejaVu Sans Mono}[Scale=MatchLowercase]

% Matematika i simboli
\usepackage{amsmath,amssymb,amsthm}
\usepackage{mathtools}
\usepackage{stmaryrd} % Za dodatne logičke simbole
\usepackage{bussproofs} % Za dokaze u prirodnoj dedukciji

% Programski kod
\usepackage{listings}
\usepackage{minted} % Naprednije označavanje koda
\usepackage{tcolorbox}
\tcbuselibrary{minted,skins,breakable}

% Grafika i boje
\usepackage{graphicx}
\usepackage{tikz}
\usetikzlibrary{trees,arrows,positioning,calc,shapes}
\usepackage{xcolor}
\usepackage{colortbl}

% Dizajn stranice
\usepackage[top=2.5cm, bottom=2.5cm, left=3cm, right=2.5cm]{geometry}
\usepackage{fancyhdr}
\usepackage{titlesec}
\usepackage{microtype} % Poboljšava tipografiju

% Hiperveze i reference
\usepackage[colorlinks=true,linkcolor=blue!70!black,citecolor=green!50!black,urlcolor=blue!70!black]{hyperref}
\usepackage{cleveref}
%\usepackage[unicode, pdfencoding=auto]{hyperref}

% Tablice i liste
\usepackage{booktabs}
\usepackage{array}
\usepackage{enumitem}
\usepackage{multicol}

% Dodatni paketi
\usepackage{epigraph}
\usepackage{marginnote}
\usepackage{mdframed}
\usepackage{caption}
\usepackage{subcaption}

% ===================================
% DEFINICIJE BOJA
% ===================================

\definecolor{pythonblue}{RGB}{53,114,165}
\definecolor{pythonyellow}{RGB}{255,217,64}
\definecolor{pythongreen}{RGB}{40,167,69}
\definecolor{pythonred}{RGB}{220,53,69}
\definecolor{pythongray}{RGB}{108,117,125}
\definecolor{codebg}{RGB}{248,249,250}
\definecolor{theorembg}{RGB}{240,248,255}
\definecolor{examplebg}{RGB}{245,255,245}
\definecolor{warningbg}{RGB}{255,250,240}

% ===================================
% STILOVI ZA KOD (BREAKABLE VERSION)
% ===================================

% Definicija Python stila za listings
\lstdefinestyle{pythonstyle}{
    language=Python,
    basicstyle=\ttfamily\small,
    keywordstyle=\color{pythonblue}\bfseries,
    stringstyle=\color{pythongreen},
    commentstyle=\color{pythongray}\itshape,
    numbers=left,
    numberstyle=\tiny\color{pythongray},
    stepnumber=1,
    numbersep=10pt,
    backgroundcolor=\color{codebg},
    frame=single,
    frameround=tttt,
    framesep=5pt,
    breaklines=true,
    breakatwhitespace=true,
    tabsize=4,
    showstringspaces=false,
    captionpos=b
}

% Stilizirana kutija za Python kod - BREAKABLE VERSION
\newtcblisting{pythoncode}[1][]{
    listing engine=minted,
    minted language=python,
    minted options={
        fontsize=\small,
        breaklines,
        autogobble,
        linenos,
        numbersep=5pt
    },
    colback=codebg,
    colframe=pythonblue!70,
    listing only,
    left=5mm,
    enhanced,
    breakable,  % Makes the box breakable across pages
    overlay={\begin{tcbclipinterior}\fill[pythonblue!20] (frame.south west) rectangle ([xshift=5mm]frame.north west);\end{tcbclipinterior}},
    #1
}

% Breakable output box using tcolorbox
\newtcolorbox{codeoutput}{
    colback=gray!10,
    colframe=gray!50,
    breakable,
    fonttitle=\bfseries,
    title=Izlaz:,
    before upper=\small\ttfamily,
    sharp corners,
    boxrule=1pt,
    left=2mm,
    right=2mm,
    top=2mm,
    bottom=2mm
}

% ===================================
% TEOREMI I OKRUŽENJA
% ===================================

\theoremstyle{definition}
\newtheorem{definicija}{Definicija}[chapter]
\newtheorem{aksiom}{Aksiom}[chapter]
\newtheorem{napomena}{Napomena}[chapter]

\theoremstyle{plain}
\newtheorem{teorem}{Teorem}[chapter]
\newtheorem{propozicija}{Propozicija}[chapter]
\newtheorem{lema}{Lema}[chapter]
\newtheorem{korolar}{Korolar}[chapter]

\theoremstyle{remark}
\newtheorem{primjer}{Primjer}[chapter]
\newtheorem{vježba}{Vježba}[chapter]

% Stilizirana okruženja
\newenvironment{definicijaokvir}
    {\begin{mdframed}[backgroundcolor=theorembg,linecolor=pythonblue,linewidth=2pt]
    \begin{definicija}}
    {\end{definicija}\end{mdframed}}

\newenvironment{primjerokvir}
    {\begin{mdframed}[backgroundcolor=examplebg,linecolor=pythongreen,linewidth=2pt]
    \begin{primjer}}
    {\end{primjer}\end{mdframed}}

\newenvironment{upozorenje}
    {\begin{mdframed}[backgroundcolor=warningbg,linecolor=pythonred,linewidth=2pt]
    \textbf{Upozorenje:} }
    {\end{mdframed}}

% ===================================
% PRILAGODBA NASLOVA
% ===================================

\titleformat{\chapter}[display]
{\normalfont\huge\bfseries\color{pythonblue}}
{\chaptertitlename\ \thechapter}{20pt}{\Huge}

\titleformat{\section}
{\normalfont\Large\bfseries\color{pythonblue!80}}
{\thesection}{1em}{}

\titleformat{\subsection}
{\normalfont\large\bfseries\color{pythonblue!60}}
{\thesubsection}{1em}{}

% ===================================
% ZAGLAVLJA I PODNOŽJA
% ===================================

\pagestyle{fancy}
\fancyhf{}
\fancyhead[LE,RO]{\thepage}
\fancyhead[LO]{\rightmark}
\fancyhead[RE]{\leftmark}
\renewcommand{\headrulewidth}{0.4pt}
\renewcommand{\footrulewidth}{0pt}

% ===================================
% MAKRONAREDBE
% ===================================

% Logički simboli prema pojmovniku
\newcommand{\logsud}[1]{\mathit{#1}} % Za sudove/propozicije
\newcommand{\konj}{\land}             % Konjunkcija
\newcommand{\disj}{\lor}              % Disjunkcija  
\newcommand{\impl}{\rightarrow}       % Implikacija
\newcommand{\biimp}{\leftrightarrow}  % Bikondicional
\newcommand{\svaki}{\forall}          % Univerzalni kvantifikator
\newcommand{\postoji}{\exists}        % Egzistencijalni kvantifikator

% Python inline kod
\newcommand{\pyinline}[1]{\texttt{\color{pythonblue}#1}}

% Jupyter bilježnica referenca
\newcommand{\jupyter}[1]{\marginpar{\small\color{pythongreen}\faFileCode\ #1.ipynb}}

% ===================================
% DOKUMENT
% ===================================

\begin{document}

% ----------------------------------
% NASLOVNICA
% ----------------------------------
\frontmatter

\begin{titlepage}
    \centering
    \vspace*{2cm}
    
    {\Huge\bfseries\color{pythonblue} LOGIKA U KODU\par}
    \vspace{1cm}
    {\Large\itshape Elementi logike kroz programski jezik Python\par}
    
    \vspace{3cm}
    
    \begin{tikzpicture}
        \node[circle,draw=pythonblue,line width=2pt,minimum size=3cm] (logic) at (0,0) {
            \Large $\vdash$
        };
        \node[circle,draw=pythonyellow,line width=2pt,minimum size=3cm] (python) at (2,0) {
            \Large \texttt{Py}
        };
        \draw[->,thick,pythongreen] (logic) -- (python);
    \end{tikzpicture}
    
    \vspace{3cm}
    
    {\large\scshape Priručnik za kolegij ,,Logika i programiranje''\par}
    \vspace{1cm}
    {\large Davor Lauc\par}
    
    \vfill
    
    {\large Filozofski Fakultet Sveučilišta u Zagrebu\par}
    {\large 2025\par}
\end{titlepage}

% ----------------------------------
% SADRŽAJ
% ----------------------------------
\tableofcontents

% ----------------------------------
% PREDGOVOR
% ----------------------------------
\chapter*{Predgovor}
\addcontentsline{toc}{chapter}{Predgovor}

Ovaj udžbenik predstavlja moderan pristup učenju logike kroz programski jezik Python, prvenstveno namijenjen polaznicima kolegija ,,Logika i programiranje'' ali i svima onima koji uče logiku ili programiranje, te žele iskoristiti transfer znanja iz jednog područja u drugi.
Spajanjem apstraktnih logičkih koncepata s konkretnom implementacijom u kodu, studenti mogu neposredno eksperimentirati s logičkim sustavima i dublje razumjeti njihovu strukturu.

Udžbenik koristi hrvatsku terminologiju prema standardnom pojmovniku logike, ali uvodi i međunarodne termine važne za rad u području računarstva i umjetne inteligencije.

Svako poglavlje popraćeno je Jupyter bilježnicama koje omogućuju interaktivno učenje i eksperimentiranje.

% ----------------------------------
% UVOD
% ----------------------------------
\mainmatter

\chapter{Uvod: Što je logika i zašto kod?}
% Modificirani uvod.tex - čisti LaTeX format


\epigraph{Ne priliči izvrsnim ljudima da kao robovi gube sate na računanje, u poslu koji se može s punim povjerenjem prepustiti bilo kome drugome uporabom strojeva.}{Gottfried Wilhelm Leibniz\footnotemark}
\footnotetext{\emph{Machina Arithmetica in qua non Aditio tantum Subtractio 1685}, slobodni prijevod s engleskog prijevoda}


\section{Logika kao znanost o valjanom zaključivanju}


Logika proučava oblike valjanog zaključivanja i relacije logičkog slijeda. Kada kažemo da je zaključak valjan, mislimo da konkluzija nužno slijedi iz premisa. Ova nužnost proizlazi iz same strukture našeg mišljenja, neovisno o sadržaju.


Razmotrimo klasičan primjer:
\begin{enumerate}
\item Svi ljudi su smrtni.
\item Sokrat je čovjek.
\item Dakle, Sokrat je smrtan.
\end{enumerate}


Valjanost ovog zaključka ne ovisi o tome tko je Sokrat ili što znači čovjek'' i smrtan''. Ona proizlazi iz logičke forme koja ostaje valjana čak i kad zamijenimo termine:
\begin{enumerate}
\item Svi $P$ su $Q$.
\item $a$ je $P$.
\item Dakle, $a$ je $Q$.
\end{enumerate}


Moderna simbolička logika omogućava nam precizno zapisivanje ovakvih formi. U logici sudova bavimo se veznicima:
\begin{itemize}
\item Konjunkcija ($\wedge$): i'' \item Disjunkcija ($\vee$): ili''

\item Implikacija ($\rightarrow$): ako...onda'' \item Negacija ($\neg$): nije''
\end{itemize}


Logika predikata ide dalje omogućavajući kvantifikaciju:
\begin{itemize}
\item Univerzalni kvantifikator ($\forall$): ,,svi'', ,,svaki''
\item Egzistencijalni kvantifikator ($\exists$): ,,neki'', ,,postoji''
\end{itemize}


\section{Formalni jezik misli}


Gottlob Frege, utemeljitelj moderne logike, nastojao je stvoriti ..pojmovno pismo'' (\textit{Begriffsschrift}) -- formalni jezik za izražavanje čistog mišljenja, oslobođenog dvosmislenosti prirodnog jezika. Njegova vizija danas živi u programskim jezicima.


Kada pišemo Python kod:
\begin{verbatim}
def je_smrtan(x):
    return je_čovjek(x)

assert je_smrtan("Sokrat") == True
\end{verbatim}


formaliziramo logičku strukturu. Funkcija \texttt{je\_smrtan} enkodira univerzalnu tvrdnju, a \texttt{assert} provjerava konkretnu instancu.


\section{Struktura stvarnosti: Wittgensteinova slika}


Ludwig Wittgenstein u \textit{Tractatus Logico-Philosophicus} nudi radikalnu tezu: granice našeg jezika su granice našeg svijeta. Za njega, svijet je totalitet činjenica, ne stvari. Činjenice postoje u ..logičkom prostoru'' -- prostoru svih mogućih kombinacija.


Ova ideja ima izravnu programsku interpretaciju. Kada definiramo Booleove varijable:
\begin{verbatim}
kiša = True
sunce = False
vjetar = True
\end{verbatim}


definiramo točku u logičkom prostoru. S tri varijable, imamo $2^3 = 8$ mogućih svjetova. Svaki program implicitno definira takav prostor mogućnosti i pravila kretanja kroz njega.


\section{Istina i značenje: Tarskijeva semantika}


Alfred Tarski riješio je drevni problem istine kroz svoju semantičku teoriju. Njegova čuvena T-shema:


..Snijeg je bijel'' je istinito ako i samo ako snijeg je bijel.


Može se činiti trivijalnom, ali razlikuje jezik od metajezika. U Pythonu:
\begin{verbatim}
def je_istinito(izjava, model):
    return eval(izjava, model)

model = {"snijeg_je_bijel": True}
assert je_istinito("snijeg_je_bijel", model) == True
\end{verbatim}


funkcija \texttt{je\_istinito} je metajezična -- ona govori \emph{o} izjavama, ne \emph{u} njima.


\section{Dokazi kao programi: Curry-Howard izomorfizam}


Jedan od najdubljih uvida 20. stoljeća je otkriće strukturne ekvivalencije između logičkih dokaza i programa. Curry-Howard izomorfizam pokazuje:


\begin{center}
\begin{tabular}{l|l}
\textbf{Logika} & \textbf{Programiranje} \\
\hline
Formula $A \rightarrow B$ & Tip funkcije \texttt{A -> B} \\
Dokaz formule & Program tog tipa \\
Modus ponens & Aplikacija funkcije \\
Konjunkcija $A \wedge B$ & Tuple \texttt{(A, B)} \\
Disjunkcija $A \vee B$ & Union type \texttt{A | B} \\
\end{tabular}
\end{center}


Svaki put kad pišete funkciju, vi zapravo konstruirate dokaz. Svaki put kad je pozivate, primjenjujete logičko pravilo. Tipovi su teoremi, programi su dokazi!


\section{Granice formalizma: Gödelova nepotpunost}


Kurt Gödel je 1931. dokazao da svaki dovoljno bogat formalni sustav ili je nekonzistentan ili nepotpun. Postoje istinite tvrdnje koje se ne mogu dokazati unutar sustava.


Njegov dokaz koristi samoreferenciranje -- konstruira rečenicu koja kaže ..Ja nisam dokaziva''. Ako je dokaziva, sustav je nekonzistentan. Ako nije, sustav je nepotpun.


U Pythonu možemo ilustrirati Gödelov trik:
\begin{verbatim}
def gödel_rečenica(n):
    """Rečenica koja tvrdi da nije dokaziva"""
    return f"Rečenica broj {n} nije dokaziva"


\textbf{Paradoks: ako je dokaziva, onda je lažna}

\textbf{Ako nije dokaziva, onda je istinita, ali nedokaziva}

\end{verbatim}


\section{Turingovi strojevi}


Alan Turing definirao je precizni model računanja -- Turingov stroj. Pokazao je da postoje problemi koje nijedan stroj ne može riješiti, poput problema zaustavljanja.


Python interpreter je zapravo Turingov stroj (tehnički Turing potpun jezik).
Svaki program koji napišete je opis konačnog automata koji manipulira simbolima na ..traci'' (memoriji):

\begin{verbatim}
def turingov_stroj(traka, program, stanje=0):
    while stanje != "STOP":
        simbol = traka.pročitaj()
        novo_stanje, novi_simbol, smjer = program[stanje][simbol]
        traka.zapiši(novi_simbol)
        traka.pomakni(smjer)
        stanje = novo_stanje
    return traka
\end{verbatim}


\section{Od dedukcije k indukciji}

Klasična logika bavi se nužnim zaključcima. Ali većina našeg zaključivanja je probabilistička. David Hume primijetio je ,,problem indukcije'' -- iz činjenice da je sunce izašlo svaki dan do sada, ne slijedi nužno da će izaći sutra.

Bayesov teorem daje nam formalni okvir za induktivno zaključivanje:

$$P(H|E) = \frac{P(E|H) \cdot P(H)}{P(E)}$$


Gdje je $P(H|E)$ posteriorna vjerojatnost hipoteze nakon evidencije.


U Pythonu:
\begin{verbatim}
def bayes(prior, likelihood, evidence):
return (likelihood * prior) / evidence


\textbf{Medicinska dijagnoza}

p_bolest = 0.01  # 1% populacije ima bolest
p_pozitivan_test_ako_bolest = 0.99  # 99% osjetljivost
p_pozitivan_test = 0.05  # 5% ukupno pozitivnih


p_bolest_ako_pozitivan = bayes(
p_bolest,
p_pozitivan_test_ako_bolest,
p_pozitivan_test
)


\textbf{Rezultat: 0.198 ili 19.8%}

\end{verbatim}


\section{Paradoksi i granice}


Logika je puna paradoksa koji testiraju naše razumijevanje:


\textbf{Russellov paradoks}: Skup svih skupova koji ne sadrže sebe. Sadrži li sebe?


\textbf{Paradoks lažljivca}: ..Ova rečenica je neistinita.'' Istinita ili neistinita?


\textbf{Sorites paradoks}: Jedna zrna pijeska nije hrpa. Dodavanje jednog zrna ne čini hrpu. Dakle, nikad nema hrpe?


Ovi paradoksi nisu samo zagonetke -- oni otkrivaju fundamentalne limite formalnih sustava i potiču razvoj novih logika (parakontistentne, fuzzy, relevantne).


\section{Primjena u stvarnom svijetu}


Logika kroz kod nije samo akademska vježba. Primjene su svugdje:


\textbf{Baze podataka} koriste logiku predikata (SQL je zapravo logički jezik):
\begin{verbatim}
SELECT * FROM studenti
WHERE godina > 2 AND prosjek >= 4.0
\end{verbatim}


\textbf{Verifikacija softvera} dokazuje korektnost kritičnih sustava:
\begin{verbatim}
@requires(x >= 0)
@ensures(rezultat >= 0)
def korijen(x):
    return sqrt(x)
\end{verbatim}


\textbf{Umjetna inteligencija} koristi logičko zaključivanje za planiranje i donošenje odluka.




\section{Putovanje koje slijedi}


Ovaj udžbenik vodi vas kroz postupno otkrivanje veze između logike i programiranja. Svaki koncept gradit ćemo od temelja:


\begin{enumerate}
\item \textbf{Intuicija}: Zašto je koncept važan?
\item \textbf{Formalizacija}: Precizna matematička definicija
\item \textbf{Implementacija}: Radni Python kod
\item \textbf{Eksploracija}: Eksperimenti i varijacije
\end{enumerate}


Ne učite samo \emph{o} logici -- prakticirajte logiku kroz kod. Svaka Jupyter bilježnica je laboratorij. Mijenjajte parametre, testirajte granične slučajeve, pokušajte pokvariti kod. Kroz ove eksperimente razvit ćete dublju intuiciju od pukog čitanja.


Zapamtite: u logici, kao i u programiranju, jedini način učenja je činjenjem. Greške su dragocjene -- one otkrivaju skrivene pretpostavke i suptilne istine.


Započnimo ovo putovanje kroz svjetove logike, gdje svaki novi koncept otvara vrata dubljeg razumijevanja načina na koji mislimo, zaključujemo i stvaramo.



% ----------------------------------
% DIO I: OSNOVE
% ----------------------------------
\part{SVJETOVI DEDUKTIVNE LOGIKE}
\input{wittgenstein}
\input{gentzen}
\input{tarski}
\input{turing}
\input{cantor}

\part{SVJETOVI INDUKTIVNIH LOGIKA}
\input{pascal}
\input{bayes}
\input{goodman}


% ----------------------------------
% DODACI
% ----------------------------------
\appendix


\chapter{Uvod u Python za studente filozofije i ostalih ne-tehničkih grupa)}
\label{chap:Uvod u Pzthon}

\section{Zašto bi se filozof zanimao za programiranje?}

Na prvi pogled, svijet filozofije i svijet programiranja mogu se činiti kao dva potpuno odvojena svemira. Jedan se bavi vječnim pitanjima o smislu, postojanju i vrijednostima, dok se drugi bavi preciznim uputama za strojeve. Međutim, ispod površine, ova dva svijeta dijele duboke i iznenađujuće veze. Logika, temeljni alat filozofske analize, ujedno je i srce svakog računalnog programa. Način na koji strukturiramo argumente, definiramo pojmove i izvodimo zaključke u filozofiji ima svoj odraz u načinu na koji pišemo kod.

Učenje programskog jezika \pyinline{Python}, stoga, za studenta filozofije nije samo stjecanje tehničke vještine, već i prilika za istraživanje poznatih koncepata iz nove perspektive. Kroz \pyinline{Python}, apstraktni pojmovi poput varijabli, uvjeta i petlji postaju konkretni alati s kojima možete raditi, eksperimentirati i stvarati.

Ovo poglavlje je osmišljeno kao blagi uvod u \pyinline{Python}, posebno prilagođen studentima humanističkih i društvenih znanosti. Nećemo se baviti složenim matematičkim problemima niti dubokim tehničkim detaljima. Umjesto toga, fokusirat ćemo se na osnove jezika, koristeći primjere koji su vam bliski: analizu teksta, rad s riječima i rečenicama, te istraživanje ideja kroz kod. Koristit ćemo se \textbf{Jupyter bilježnicama}, interaktivnim okruženjem koje omogućuje pisanje koda, teksta i vizualizacija na jednom mjestu, čineći učenje intuitivnim i zabavnim.

Dok budete prolazili kroz ovo poglavlje, potičem vas da ne gledate na kod samo kao na niz naredbi, već kao na novi način izražavanja i strukturiranja misli. Možda ćete otkriti da vam učenje programiranja može pomoći da postanete precizniji u svom filozofskom promišljanju, jasniji u svom izražavanju i kreativniji u svom pristupu problemima. Dobrodošli u svijet \pyinline{Python}a!

\section{Osnovni pojmovi: Varijable, tipovi podataka i izrazi}
\label{sec:osnovnipojmovi}

\subsection{Varijable: Imenovanje ideja}

U filozofiji, često koristimo simbole ili nazive kako bismo predstavili složene ideje. Na primjer, u logici, slovo \logsud{P} može predstavljati propoziciju "Svi ljudi su smrtni". Na sličan način, u \pyinline{Python}u koristimo \textbf{varijable} kao imenovane spremnike za pohranu podataka.

\begin{definicijaokvir}
    \textbf{Varijabla} je imenovani prostor u memoriji koji služi za pohranu vrijednosti. Ime varijable (identifikator) koristimo kako bismo pristupili pohranjenoj vrijednosti.
\end{definicijaokvir}

Varijablu možete zamisliti kao oznaku koju pridružujete nekoj vrijednosti. Operator dodjele, znak jednakosti (\pyinline{=}), koristi se za dodjeljivanje vrijednosti varijabli.

\begin{primjerokvir}
    Dodjeljivanje vrijednosti varijablama.
    \begin{pythoncode}
pozdrav = "Zdravo, svijete!"
godinarodenjakanta = 1724
pipriblizno = 3.14159
    \end{pythoncode}
    U ovom primjeru, \pyinline{pozdrav}, \pyinline{godinarodenjakanta} i \pyinline{pipriblizno} su nazivi varijabli. Jednom kada definiramo varijablu, možemo je koristiti u daljnjem kodu, na primjer za ispis njezine vrijednosti pomoću ugrađene funkcije \pyinline{print()}.
    \begin{pythoncode}
print(pozdrav)
print(godinarodenjakanta)
    \end{pythoncode}
    \begin{codeoutput}
Zdravo, svijete!
1724
    \end{codeoutput}
\end{primjerokvir}


\subsection{Tipovi podataka: Različite vrste informacija}

U filozofiji, razlikujemo različite vrste pojmova: konkretne, apstraktne, pojedinačne, opće. Slično tome, u \pyinline{Python}u, svaka vrijednost pripada određenom \textbf{tipu podataka}. Osnovni tipovi podataka koje ćemo za početak koristiti su:
\begin{itemize}[leftmargin=*]
    \item \textbf{String (\pyinline{str}):} Niz znakova, odnosno tekstualni podaci. Stringovi se uvijek pišu unutar navodnika (jednostrukih \pyinline{''} ili dvostrukih \pyinline{""}). Primjeri: \pyinline{"Sokrat"}, \pyinline{'Platonova Država'}.
    \item \textbf{Integer (\pyinline{int}):} Cijeli brojevi, bez decimalnog dijela. Primjeri: \pyinline{42}, \pyinline{-399}, \pyinline{2025}.
    \item \textbf{Float (\pyinline{float}):} Brojevi s pomičnim zarezom (decimalni brojevi). Primjeri: \pyinline{3.14}, \pyinline{9.81}, \pyinline{-0.5}.
    \item \textbf{Boolean (\pyinline{bool}):} Logička ili istinitosna vrijednost. Može imati samo dvije vrijednosti: \pyinline{True} (istina) ili \pyinline{False} (laž).
\end{itemize}

\pyinline{Python} je dinamički tipiziran jezik, što znači da ne moramo unaprijed deklarirati tip varijable. Interpretator automatski prepoznaje tip podatka kada dodijelimo vrijednost. Tip varijable možemo provjeriti pomoću ugrađene funkcije \pyinline{type()}.

\begin{primjerokvir}
    Provjera tipova podataka.
    \begin{pythoncode}
filozof = "Aristotel"
godinarodenja = -384
visinaumetrima = 1.7
jeliziv = False

print(type(filozof))
print(type(godinarodenja))
print(type(visinaumetrima))
print(type(jeliziv))
    \end{pythoncode}
    \begin{codeoutput}
<class 'str'>
<class 'int'>
<class 'float'>
<class 'bool'>
    \end{codeoutput}
\end{primjerokvir}

\subsection{Izrazi: Kombiniranje vrijednosti}

U logici, kombiniramo propozicije pomoću veznika ($\konj, \disj, \impl$) kako bismo stvorili složenije izraze. U \pyinline{Python}u, \textbf{izrazi} su kombinacije vrijednosti, varijabli i operatora koje se izračunavaju (evaluiraju) kako bi proizvele novu vrijednost.

\begin{itemize}[leftmargin=*]
    \item \textbf{Aritmetički izrazi:} Koriste standardne matematičke operatore (\pyinline{+}, \pyinline{-}, \pyinline{*}, \pyinline{/}).
    \begin{pythoncode}
a = 10
b = 5
zbroj = a + b
print(zbroj)
    \end{pythoncode}
    \begin{codeoutput}
15
    \end{codeoutput}

    \item \textbf{String izrazi:} Operator \pyinline{+} se može koristiti za spajanje (\textit{konkatenaciju}) stringova.
    \begin{pythoncode}
ime = "Immanuel"
prezime = "Kant"
punoime = ime + " " + prezime
print(punoime)
    \end{pythoncode}
    \begin{codeoutput}
Immanuel Kant
    \end{codeoutput}
\end{itemize}

\section{Strukture podataka: Organiziranje misli}
\label{sec:strukturepodataka}

Dok osnovni tipovi podataka predstavljaju pojedinačne vrijednosti, \textbf{strukture podataka} služe za organiziranje i pohranu više vrijednosti u jednoj varijabli.

\subsection{Liste: Uređeni nizovi argumenata}

U filozofskim tekstovima, često nailazimo na nabrajanja ili nizove ideja, poput Aristotelovih četiriju uzroka. U \pyinline{Python}u, \textbf{liste} su strukture podataka koje nam omogućuju pohranu uređenog niza elemenata. Elementi liste se navode unutar uglatih zagrada \pyinline{[]}, odvojeni zarezima.

\begin{definicijaokvir}
    \textbf{Lista} (\pyinline{list}) je promjenjiva, uređena kolekcija elemenata. "Uređena" znači da elementi zadržavaju redoslijed kojim su dodani. "Promjenjiva" znači da možemo dodavati, uklanjati ili mijenjati elemente nakon što je lista stvorena.
\end{definicijaokvir}

\begin{primjerokvir}
    Kreiranje i pristupanje elementima liste.
    \begin{pythoncode}

## Lista Aristotelovih uzroka
aristoteloviuzroci = ["materijalni", "formalni", "djelatni", "svršni"]

# Pristupanje elementima pomoću indeksa
# Indeksiranje počinje od 0!
prviuzrok = aristoteloviuzroci[0]
treciuzrok = aristoteloviuzroci[2]

print("Prvi uzrok je:", prviuzrok)
print("Treći uzrok je:", treciuzrok)
    \end{pythoncode}
    \begin{codeoutput}
Prvi uzrok je: materijalni
Treći uzrok je: djelatni
    \end{codeoutput}

    Liste su promjenjive. Možemo im dodavati nove elemente metodom \pyinline{append()} ili uklanjati postojeće metodom \pyinline{remove()}.

    \begin{pythoncode}
# Dodavanje petog, "imaginarnog" uzroka
aristoteloviuzroci.append("imaginarni")
print(aristoteloviuzroci)

# Uklanjanje "imaginarnog" uzroka
aristoteloviuzroci.remove("imaginarni")
print(aristoteloviuzroci)
    \end{pythoncode}
    \begin{codeoutput}
['materijalni', 'formalni', 'djelatni', 'svršni', 'imaginarni']
['materijalni', 'formalni', 'djelatni', 'svršni']
    \end{codeoutput}
\end{primjerokvir}

\subsection{Rječnici: Asocijativni parovi pojmova i definicija}

U filozofiji, često definiramo pojmove tako da im pridružujemo njihove definicije. U \pyinline{Python}u, \textbf{rječnici} (\pyinline{dict}) omogućuju pohranu podataka u obliku parova \textbf{ključ-vrijednost}. Ključevi su jedinstveni i koriste se za pristup pripadajućim vrijednostima.

\begin{definicijaokvir}
    \textbf{Rječnik} (\pyinline{dict}) je promjenjiva, neuređena kolekcija parova ključ-vrijednost. Svaki ključ mora biti jedinstven unutar rječnika.
\end{definicijaokvir}

Rječnici se definiraju unutar vitičastih zagrada \pyinline{{}}, a parovi ključ-vrijednost odvojeni su dvotočkom.

\begin{primjerokvir}
    Kreiranje i korištenje rječnika.
    \begin{pythoncode}
filozofskirjecnik = {
    "epistemologija": "grana filozofije koja se bavi znanjem",
    "metafizika": "grana filozofije koja se bavi prvim uzrocima i principima bića",
    "etika": "grana filozofije koja se bavi moralom"
}

# Pristupanje vrijednosti pomoću ključa
definicijaetike = filozofskirjecnik["etika"]
print(definicijaetike)

# Dodavanje novog para
filozofskirjecnik["logika"] = "znanost o metodama i principima ispravnog zaključivanja"
print(filozofskirjecnik["logika"])
    \end{pythoncode}
    \begin{codeoutput}
grana filozofije koja se bavi moralom
znanost o metodama i principima ispravnog zaključivanja
    \end{codeoutput}
\end{primjerokvir}

\section{Kontrola toka: Usmjeravanje argumentacije}
\label{sec:kontrolatoka}

Programi se ne izvršavaju uvijek linearno, od prve do zadnje naredbe. \textbf{Kontrola toka} odnosi se na naredbe koje nam omogućuju da usmjeravamo tijek izvršavanja programa, donosimo odluke i ponavljamo operacije.

\subsection{Uvjetno izvršavanje: \pyinline{if}, \pyinline{elif}, \pyinline{else}}

U filozofskoj argumentaciji, često koristimo uvjetne rečenice oblika "Ako $\logsud{P}$, onda $\logsud{Q}$". U \pyinline{Python}u, \textbf{uvjetne naredbe} nam omogućuju da izvršimo određeni dio koda samo ako je zadovoljen neki uvjet. Uvjet je izraz koji se evaluira kao \pyinline{True} ili \pyinline{False}.

\begin{primjerokvir}
    Korištenje \pyinline{if-else} strukture.
    \begin{pythoncode}
tvrdnja = "Sokrat je smrtan"

if "Sokrat" in tvrdnja:
    print("Tvrdnja se odnosi na Sokrata.")
else:
    print("Tvrdnja se ne odnosi na Sokrata.")
    \end{pythoncode}
    \begin{codeoutput}
Tvrdnja se odnosi na Sokrata.
    \end{codeoutput}

    Možemo koristiti i \pyinline{elif} (skraćeno od \textit{else if}) za provjeru više uzastopnih uvjeta.

    \begin{pythoncode}
godina = 1804

if godina < 476:
    print("Antička filozofija")
elif 476 <= godina < 1500:
    print("Srednjovjekovna filozofija")
else:
    print("Moderna i suvremena filozofija")
    \end{codeoutput}
    \begin{codeoutput}
Moderna i suvremena filozofija
    \end{codeoutput}
\end{primjerokvir}

\subsection{Ponavljanje: \pyinline{for} petlja}

Često je potrebno ponoviti istu radnju više puta. Na primjer, analizirati svaku riječ u rečenici. U \pyinline{Python}u, \textbf{\pyinline{for} petlja} nam omogućuje da iteriramo (prolazimo) kroz elemente sekvence (poput liste ili stringa) i za svaki element izvršimo određeni blok koda.

\begin{primjerokvir}
    Iteriranje kroz listu pomoću \pyinline{for} petlje.
    \begin{pythoncode}
stoickevrline = ["mudrost", "pravednost", "hrabrost", "umjerenost"]

print("Prema stoicima, temeljne vrline su:")
for vrlina in stoickevrline:

    print("- " + vrlina)
    \end{pythoncode}
    \begin{codeoutput}
Prema stoicima, temeljne vrline su:
- mudrost
- pravednost
- hrabrost
- umjerenost
    \end{codeoutput}

    U ovom primjeru, varijabla \pyinline{vrlina} se naziva \textit{varijabla petlje}. U svakom prolasku (iteraciji) kroz petlju, ona poprima vrijednost sljedećeg elementa iz liste \pyinline{stoickevrline}.
\end{primjerokvir}

\section{Funkcije: Modularizacija i ponovna upotreba misli}
\label{sec:funkcije}

U filozofiji, kompleksne ideje često razlažemo na manje, razumljivije dijelove. U \pyinline{Python}u, \textbf{funkcije} nam omogućuju da grupiramo niz naredbi u logičku cjelinu koju možemo pozvati više puta. Time se izbjegava ponavljanje koda i programi postaju organiziraniji i lakši za čitanje.

\begin{definicijaokvir}
    \textbf{Funkcija} je imenovani blok koda koji izvršava određeni zadatak. Može primiti ulazne podatke (argumente) i vratiti izlaznu vrijednost.
\end{definicijaokvir}

Funkcije definiramo pomoću ključne riječi \pyinline{def}.

\begin{primjerokvir}
    Definiranje i pozivanje jednostavne funkcije.
    \begin{pythoncode}
def pozdravifilozofa(ime):
    """
    Ova funkcija ispisuje pozdrav filozofu čije je ime
    prolijeđeno kao argument.
    """
    print("Pozdrav, " + ime + "!")

# Pozivanje funkcije
pozdravifilozofa("Platon")
pozdravifilozofa("Nietzsche")
    \end{pythoncode}
    \begin{codeoutput}
Pozdrav, Platon!
Pozdrav, Nietzsche!
    \end{codeoutput}
    Tekst unutar trostrukih navodnika odmah nakon definicije funkcije naziva se \textit{docstring} i služi kao dokumentacija funkcije.
\end{primjerokvir}

Funkcije mogu vraćati vrijednost pomoću naredbe \pyinline{return}. Vraćena vrijednost se tada može pohraniti u varijablu ili koristiti u daljnjim izrazima.

\begin{primjerokvir}
    Funkcija koja vraća vrijednost.
    \begin{pythoncode}
def sastaviime(ime, prezime):
    """Sastavlja puno ime iz dva dijela."""
    return ime + " " + prezime

punoimefilozofa = sastaviime("Simone", "de Beauvoir")
print(punoimefilozofa)
    \end{pythoncode}
    \begin{codeoutput}
Simone de Beauvoir
    \end{codeoutput}
\end{primjerokvir}

\section{Primjer iz prakse: Analiza filozofskog teksta}
\label{sec:primjerprakse}

Sada ćemo primijeniti sve što smo naučili na konkretnom primjeru: analizi kratkog filozofskog teksta. Cilj nam je prebrojati koliko se puta svaka riječ pojavljuje u poznatoj Descartesovoj izreci.

\begin{primjerokvir}
    Brojanje riječi u tekstu.
    \begin{pythoncode}
# Korak 1: Definiramo tekst za analizu
tekst = "Mislim, dakle jesam. Jesam, dakle postojim."
print("Originalni tekst:", tekst)

# Korak 2: Priprema teksta
# Pretvaramo sva slova u mala slova kako 'Mislim' i 'mislim' ne bi bili različite riječi
tekstmali = tekst.lower()
# Uklanjamo interpunkcijske znakove
tekstbeztocke = tekstmali.replace('.', '')
tekstcisti = tekstbeztocke.replace(',', '')
print("Očišćeni tekst:", tekstcisti)

# Korak 3: Tokenizacija - razdvajanje teksta u listu riječi
rijeci = tekstcisti.split()
print("Lista riječi:", rijeci)

# Korak 4: Brojanje riječi pomoću rječnika
brojacrijeci = {}
for rijec in rijeci:
    if rijec in brojacrijeci:
        # Ako riječ već postoji u rječniku, povećaj brojač za 1
        brojacrijeci[rijec] = brojacrijeci[rijec] + 1
    else:
        # Ako je ovo prvo pojavljivanje riječi, dodaj je u rječnik s vrijednošću 1
        brojacrijeci[rijec] = 1

# Korak 5: Ispis rezultata
print("\nFrekvencija riječi:")
for rijec, broj in brojacrijeci.items():
    print(f"'{rijec}': {broj}")
    \end{pythoncode}

    \begin{codeoutput}
Originalni tekst: Mislim, dakle jesam. Jesam, dakle postojim.
Očišćeni tekst: mislim dakle jesam jesam dakle postojim
Lista riječi: ['mislim', 'dakle', 'jesam', 'jesam', 'dakle', 'postojim']

Frekvencija riječi:
'mislim': 1
'dakle': 2
'jesam': 2
'postojim': 1
    \end{codeoutput}
    Ovaj primjer integrira varijable, stringove i njihove metode (\pyinline{.lower()}, \pyinline{.replace()}, \pyinline{.split()}), liste, rječnike, \pyinline{for} petlju i \pyinline{if-else} uvjetnu logiku kako bi se riješio konkretan problem iz domene analize teksta.
\end{primjerokvir}

\section{Zaključak: Sljedeći koraci}
\label{sec:zakljucak}

Ovo poglavlje pružilo je kratak pregled osnovnih elemenata programskog jezika \pyinline{Python}. Vidjeli smo kako varijable i tipovi podataka predstavljaju osnovne gradivne blokove, kako strukture podataka poput lista i rječnika organiziraju informacije, kako kontrola toka usmjerava izvršavanje programa i kako funkcije omogućuju modularnost i ponovnu upotrebu koda.

Kao studenti filozofije, sada imate temelj za daljnje istraživanje. Sljedeći koraci mogli bi uključivati:
\begin{itemize}
    \item \textbf{Rad s tekstom:} \pyinline{Python} je izuzetno moćan za analizu teksta. Možete istražiti kako brojati riječi, analizirati sentiment, tražiti određene pojmove u velikim tekstualnim korpusima (npr. djelima pojedinih filozofa) i još mnogo toga. Biblioteke poput \pyinline{NLTK} (Natural Language Toolkit) i \pyinline{spaCy} otvaraju vrata svijeta \textit{računalne lingvistike}.
    \item \textbf{Vizualizacija podataka:} Pomoću biblioteka kao što su \pyinline{Matplotlib} i \pyinline{Seaborn}, možete vizualizirati odnose između pojmova, učestalost riječi ili druge uvide koje dobijete analizom teksta, pretvarajući apstraktne podatke u jasne grafove.
    \item \textbf{Web scraping:} Možete naučiti kako automatski prikupljati tekstualne podatke s web stranica, na primjer, s filozofskih enciklopedija ili online arhiva.
\end{itemize}

Najvažnije je da se ne bojite eksperimentirati. Jupyter bilježnice su idealno okruženje za to. Pokušajte mijenjati primjere, postavljati si vlastite male probleme i tražiti rješenja. Programiranje, kao i filozofija, je vještina koja se razvija kroz praksu, znatiželju i upornost. Sretno s kodiranjem!

\chapter*{Vježbe za poglavlje \ref{chap:uvod}}
\addcontentsline{toc}{chapter}{Vježbe za poglavlje \thechapter}

\begin{vježba}
    Kreirajte rječnik koji sadrži pet vaših omiljenih filozofa kao ključeve, a njihove glavne filozofske ideje ili djela kao vrijednosti. Zatim, koristeći \pyinline{for} petlju, ispišite svakog filozofa i njegovu ideju u formatu: \texttt{Ime Filozofa: Glavna ideja}.
\end{vježba}

\begin{vježba}
    Napišite funkciju pod nazivom \pyinline{brojrijeci} koja prima jedan argument (string) i vraća broj riječi u tom stringu. (Savjet: metoda \pyinline{split()} bi mogla biti korisna). Testirajte funkciju s nekoliko rečenica.
\end{vježba}

\begin{vježba}
    Napišite program koji provjerava pripada li godina određenom filozofskom razdoblju.
    \begin{enumerate}
        \item Definirajte listu koja sadrži nekoliko filozofa egzistencijalizma, npr. \pyinline{egzistencijalisti = ["Sartre", "Camus", "Kierkegaard"]}.
        \item Pitajte korisnika da unese ime filozofa pomoću funkcije \pyinline{input()}.
        \item Koristeći \pyinline{if} naredbu i operator \pyinline{in}, provjerite nalazi li se uneseno ime u vašoj listi.
        \item Ispišite odgovarajuću poruku, npr. \texttt{Sartre je egzistencijalist.} ili \texttt{Platon nije egzistencijalist.}.
    \end{enumerate}
\end{vježba}


% ----------------------------------
% BIBLIOGRAFIJA I KAZALO
% ----------------------------------
\backmatter

% \bibliographystyle{plain}
% \bibliography{literatura}


\end{document}