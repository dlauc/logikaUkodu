\chapter*{Predgovor}
\addcontentsline{toc}{chapter}{Predgovor}

Ovaj udžbenik rođen je iz uvjerenja da logika i programiranje nisu samo paralelne discipline,
već duboko isprepleteni načini mišljenja koji se međusobno osvjetljavaju.
Kroz desetljeće predavanja kolegija ,,Logika i programiranje'' na Filozofskom fakultetu Sveučilišta u Zagrebu,
svjedočio sam transformativnoj moći ovog spoja – kako studenti filozofije otkrivaju konkretnost apstraktnih
logičkih principa kroz kod, a studenti informatike pronalaze dublju strukturu svojih programa kroz formalnu logiku.

\section*{Zašto logika kroz kod?}

Gottfried Wilhelm Leibniz, čiji citat otvara ovu knjigu, sanjao je o \textit{calculus ratiocinator} – univerzalnom logičkom računu koji bi mogao riješiti sve filozofske sporove. Njegov san danas živi u programskim jezicima. Kada pišemo Python kod, mi zapravo formaliziramo misao. Svaki \texttt{if-else} izraz je logička implikacija, svaka petlja kvantifikacija, svaka funkcija dokaz.

Tradicionalni pristup učenju logike često ostaje na razini apstraktnih formula i tablica istinitosti. Ovaj udžbenik ide korak dalje – omogućava vam da \textit{pokrenete} logiku. Umjesto da samo čitate o modus ponensu, vi ga implementirate i testirate. Umjesto da memorizirate De Morganove zakone, vi ih verificirate kroz tisuće automatski generiranih primjera. Ova neposrednost iskustva transformira učenje iz pasivnog usvajanja u aktivno istraživanje.

\section*{Filozofska pozadina}

Veza između logike i računarstva nije slučajna ni površna. Curry-Howard izomorfizam, jedan od najdubljih rezultata 20. stoljeća, pokazuje da su dokazi i programi zapravo dvije strane iste medalje. Kada pišete funkciju tipa \texttt{A -> B}, vi tvrdite da možete konstruirati dokaz za B iz dokaza za A. Izvršavanje programa \textit{jest} normalizacija dokaza.

Wittgensteinov \textit{Tractatus}, koji inspirira jedno od ključnih poglavlja, nudi sliku svijeta kao totaliteta činjenica u logičkom prostoru. Naša Python implementacija ovog koncepta nije samo pedagogijska vježba – ona demonstrira kako računarski modeli mogu učiniti metafizičke koncepte operacionalnima i provjerljivima.

\section*{Struktura i pristup}

Udžbenik je organiziran u dva glavna dijela koji reflektiraju temeljnu podjelu u logici:

\textbf{Prvi dio – Svjetovi deduktivne logike} vodi vas kroz klasične sustave gdje zaključci nužno slijede iz premisa. Počinjemo s Wittgensteinovom semantičkom perspektivom, prelazimo na Gentzenovu prirodnu dedukciju, istražujemo Tarskog teoriju istine, te završavamo s Turingovim strojevima i Cantorovim dijagonalnim argumentom. Svaki sustav implementiran je u Pythonu, omogućavajući vam da eksperimentirate s različitim pristupima formalnom zaključivanju.

\textbf{Drugi dio – Svjetovi induktivnih logika} istražuje sustave gdje zaključci imaju samo određeni stupanj vjerojatnosti. Pascalova oklada uvodi nas u teoriju odlučivanja, Bayesov teorem u probabilističko zaključivanje, a Goodmanov paradoks u probleme indukcije. Ovi sustavi posebno su relevantni za moderno strojno učenje i umjetnu inteligenciju.

\section*{Pedagoški pristup}

Svako poglavlje slijedi konzistentnu strukturu: filozofska motivacija, formalna definicija,
Python implementacija, te zadaci za samostalno istraživanje.
Jupyter bilježnice omogućavaju vam da mijenjate kod, testirate granične slučajeve i razvijate vlastite eksperimente. Greške nisu prepreke već prilike – kada program ne radi kako očekujete, vi otkrivate suptilnosti logičkih principa.

Terminologija slijedi hrvatski pojmovnik logike, ali uvodi i međunarodne termine nužne za rad u globalnoj akademskoj i profesionalnoj zajednici. Kad god je moguće, dajemo etimologiju i filozofsku pozadinu termina, pomažući vam da razvijete ne samo tehničku kompetenciju već i konceptualno razumijevanje.

\section*{Za koga je ova knjiga?}

Primarno, udžbenik je namijenjen studentima filozofije koji žele razumjeti formalnu logiku kroz praktičnu primjenu.
Ali jednako je vrijedan za:

\begin{itemize}
\item Studente računarstva koji žele dublje razumjeti teoretske temelje svoje discipline
\item Istraživače u umjetnoj inteligenciji koji rade s logičkim sustavima
\item Nastavnike koji traže moderne pristupe predavanju logike
\item Autodidakte fascinirane vezom između mišljenja i računanja
\end{itemize}

Ne pretpostavljamo prethodno znanje programiranja – Python uvod u dodatku pruža sve potrebne osnove.
Također ne pretpostavljamo formalno obrazovanje iz logike – gradimo od temelja prema naprijed.

\section*{Kako koristiti ovaj udžbenik}

Materijali su dizajnirani za fleksibilno korištenje. Možete:

\begin{enumerate}
\item Pratiti linearno kroz poglavlja za sistematski kurs
\item Birati teme prema interesu za samostalno učenje
\item Koristiti kao referencu za specifične logičke sustave
\item Adaptirati Jupyter bilježnice za vlastite projekte
\end{enumerate}

Sav kod dostupan je na \texttt{https://github.com/dlauc/logikaukodu} pod Creative Commons licencom. Potičemo vas da eksperimentirate, modificirate i dijelite svoje uvide s zajednicom.

\section*{Zahvale}

Ovaj udžbenik ne bi bio moguć bez podrške mnogih. Zahvaljujem kolegama s Odsjeka za filozofiju, posebno prof. dr. sc. Zdravku Dovedanu Han i doc. dr. sc. Ines Skelac na recenzijama i vrijednim sugestijama. Studentima koji su kroz godine svojim pitanjima, nedoumicama i entuzijazom oblikovali ovaj materijal. Mojoj obitelji na strpljenju tijekom dugih sati pisanja i kodiranja.

Posebna zahvala zajednici otvorenog koda – developerima Pythona, Jupytera, i brojnih biblioteka koje omogućavaju ovu vrstu pedagoškog eksperimenta. Njihov rad pokazuje da je znanje najmoćnije kada je dijeljeno.

\section*{Poziv na putovanje}

Logika nije samo akademska disciplina – ona je način razumijevanja strukture stvarnosti i mišljenja. Programiranje nije samo praktična vještina – ono je realizacija Leibnizovog sna o formalizaciji razuma. Zajedno, oni nude jedinstvenu perspektivu na vječna pitanja: Što znači misliti? Kako možemo biti sigurni u svoje zaključke? Može li stroj razumjeti?

Pozivam vas da kroz ovu knjigu ne samo učite o logici i programiranju, već da aktivno istražujete, eksperimentirate i otkrivate. Svaka greška je lekcija, svaki uspješno pokrenut program mali trijumf razumijevanja. Neka vam ovo putovanje kroz logiku i kod bude intelektualna avantura koja mijenja način na koji vidite i svijet i vlastito mišljenje.

\vspace{1cm}
\begin{flushright}
\textit{Davor Lauc}\\
\textit{Zagreb, siječanj 2025}
\end{flushright}