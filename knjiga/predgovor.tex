\chapter*{Predgovor}
\addcontentsline{toc}{chapter}{Predgovor}

Ovaj udžbenik nastao je na temelju višegodišnjih predavanja na kolegiju „Logika i programiranje" na Filozofskom fakultetu
Sveučilišta u Zagrebu.
Kroz godine rada sa studentima filozofije koji su po prvi put pisali kod,
kao i sa studentima informatike koji su otkrivali filozofske temelje svojih programa,
oblikovao se  pristup koji spaja apstraktno i konkretno, teoriju i praksu.

Početna ideja bila je jednostavna: učiniti formalnu logiku pristupačnijom i zanimljivijom, omogućiti transfer vještina između formalne logike i programskih vještina,
te pružiti studentima filozofije dodatne ,,zapošljive'' vještine.
Umjesto da studenti samo vježbaju izvode prirodne dedukcije, zašto ih ne bi implementirali i tako bolje razumjeli?
Umjesto da crtaju istinosne tablice na papiru, zašto ih ne bi generirali programski?
Ono što je počelo kao istraživanje, pretvorilo se u potpuno novi način podučavanja logike.

Knjiga je organizirana u dva komplementarna dijela:

\textbf{Prvi dio – Svjetovi deduktivne logike} pokriva klasične sustave gdje zaključci nužno slijede iz premisa. Počinjemo s Wittgensteinovom slikom svijeta kao skupa činjenica, prelazimo na Gentzenove sustave prirodne dedukcije, istražujemo Tarskijevu semantiku, te završavamo s računalnim aspektima kroz Turingove strojeve.

\textbf{Drugi dio – Svjetovi induktivnih logika} istražuje sustave gdje zaključci imaju samo određeni stupanj vjerojatnosti: od Pascalove oklade preko Bayesova teorema do suvremenih pristupa u strojnom učenju.

Svako poglavlje slijedi konzistentnu strukturu: motivacija, formalizacija, implementacija, istraživanje.
Ova četverodijelna struktura omogućava različite razine angažmana – od prvotnog upoznavanja do dubinskog istraživanja.
Neki djelovi se djelomično preklapaju radi održavanja cjeline studenskih izlaganja.


Kao pedagoški pristup, kroz godine predavanja, razvijeno je nekoliko ključnih principa:

\textbf{Greške su pedagoški momenti.} Kada studentov kod ne radi, to nije neuspjeh već prilika za razumijevanje zašto logička pravila funkcioniraju kako funkcioniraju.

\textbf{Apstrakcija kroz konkretno.} Svaki apstraktni koncept ima konkretnu implementaciju koju je moguće pokrenuti,
modificirati i igrati se s njom.

\textbf{Spiralno učenje.} Isti koncepti vraćaju se na različitim razinama složenosti. Implikacija se prvo pojavljuje kao Python \texttt{if-then}, zatim kao materijalna implikacija,
pa kao pravilo u prirodnoj dedukciji, i konačno kao tip funkcije, i metalogički odnos.


Na pitanje za koga je ova knjiga, odgovor je da je primarno nastala za studente filozofije koji žele razumjeti formalnu logiku kroz praktičnu primjenu.
Ali kroz godine, publika se proširila:

- Studenti računarstva pronalaze filozofske temelje svoje discipline
- Nastavnici srednji škola koriste materijale za modernizaciju nastave
- Istraživači u AI-ju pronalaze korisne implementacije klasičnih sustava
- Entuzijasti koji uživaju u samostalnom istraživanju

Ne pretpostavlja se prethodno znanje programiranja – dodatak A pruža sve potrebne osnove Pythona.
Također ne pretpostavlja se formalno obrazovanje iz logike – počinje se od osnova.


Kako koristiti materijale: sav kod dostupan je na \texttt{https://github.com/dlauc/logikaukodu}. Bilježnice možete:
- Pokretati lokalno s Jupyter instalacijom
- Koristiti kroz Google Colab bez instalacije
- Modificirati za vlastite potrebe
- Integrirati u vlastite kolegije

Licenca \em{Creative Commons} omogućava slobodno dijeljenje i adaptaciju uz navođenje izvora.


Konačno, ovaj udžbenik nije završen proizvod. Svaki semestar donosi nove uvide, nove načine objašnjavanja,
nove veze između koncepata.
Pozivam čitatelje da se priključe korištenjem dijeljenog koda – prijavite greške, predložite poboljšanja, podijelite svoje implementacije.
Logika i programiranje nisu samo akademske discipline – oni su načini mišljenja koji oblikuju našu digitalnu stvarnost.
Nadam se da će ova knjiga pomoći novim generacijama da ovladaju oboma.

\vspace{1cm}
\begin{flushright}
\textit{Davor Lauc}\\
\textit{Zagreb, rujan 2025}
\end{flushright}
