
\chapter{Uvod: Što je logika i zašto kod?}

\epigraph{Ne priliči izvrsnim ljudima da kao robovi gube sate na računanje, koji se posao može sa punim povjerenjem prepustiti bilo kome drugome uporabom strojeva.}{Gottfried Wilhelm Leibniz}\footnote{\emph{Machina Arithmetica in qua non Aditio tantum Subtractio 1685}, slobodni prijevod s engleskog prijevoda}

\section*{Logika kao znanost o valjanom zaključivanju}

Logika je znanost koja proučava oblike valjanog zaključivanja i relacije logičkog slijeda. Kada kažemo da je zaključak valjan, mislimo da konkluzija nužno slijedi iz premisa. Ova nužnost nije slučajna – ona proizlazi iz same strukture našeg mišljenja.

Razmotrimo jednostavan primjer. Ako znamo da ``Svi programi imaju greške'' i da je ``Python program'', možemo zaključiti da ``Python ima greške''. Ovo zaključivanje ne ovisi o tome što mislimo o programima ili Pythonu -- ono je valjano zbog svoje logičke forme.

U suvremenoj simboličkoj logici, ovu formu možemo precizno zapisati. Logika sudova bavi se osnovnim veznicima poput konjunkcije ($\wedge$), disjunkcije ($\vee$) i implikacije ($\rightarrow$). Logika predikata ide korak dalje i omogućava nam kvantificiranje -- govorenje o ``svim'' objektima ($\forall$) ili ``nekim'' objektima ($\exists$).

\section*{Python kao jezik za istraživanje logičkih principa}

Zašto baš Python? Ovaj programski jezik nudi idealnu ravnotežu između jednostavnosti i moći. Njegova sintaksa je bliska prirodnom jeziku, što omogućava lakše razumijevanje logičkih koncepata. Istovremeno, Python je dovoljno moćan da implementira složene logičke sustave.

Kada u Pythonu pišemo:
\begin{verbatim}
if pada_kiša and nemam_kišobran:
    postat_ću_mokar = True
\end{verbatim}

Mi zapravo koristimo logičku konjunkciju. Uvjet \texttt{pada\_kiša and nemam\_kišobran} je istinit samo ako su oba dijela istinita -- upravo kao logički veznik $\wedge$.

Python nam omogućava da logičke koncepte učinimo opipljivima. Možemo implementirati istinosne tablice, provjeriti valjanost zaključaka i vizualizirati logičke strukture. Kroz kod, apstraktni logički principi postaju konkretni i provjerljivi.

\section*{Filozofska pozadina i praktična primjena}

Veza između logike i računarstva nije slučajna. Oba područja bave se preciznim opisivanjem i manipuliranjem informacija. Gottlob Frege, utemeljitelj moderne logike, želio je stvoriti ``pojmovno pismo'' (\textit{Begriffsschrift}) -- formalni jezik za izražavanje čistog mišljenja. Danas, programski jezici ostvaruju upravo tu viziju.

Ludwig Wittgenstein u svom \textit{Tractatusu} tvrdi da logika pokazuje strukturu stvarnosti. Kada pišemo program, mi zapravo opisujemo moguće svjetove -- kombinacije stanja koje naš sustav može poprimiti. Svaka \texttt{if-else} struktura definira grananje između mogućih svjetova.

Praktične primjene ove veze su svugdje oko nas. Sustavi za automatsko zaključivanje pomažu u medicinskoj dijagnostici. Formalna verifikacija osigurava ispravnost kritičnih sustava poput autopilota u zrakoplovima. Baze podataka koriste logiku predikata za učinkovito pretraživanje informacija.

\section*{Struktura i uporaba Jupyter bilježnica}

Jupyter bilježnice omogućavaju nam jedinstveni pristup učenju logike. U istom dokumentu možemo kombinirati:
\begin{itemize}
\item LaTeX formule za precizno zapisivanje logičkih izraza
\item Python kod za implementaciju i testiranje koncepata
\item Vizualizacije koje čine apstraktne ideje jasnima
\item Objašnjenja koja povezuju teoriju s praksom
\end{itemize}

Ovaj pristup omogućava interaktivno istraživanje. Umjesto pasivnog čitanja o modus ponensu, možete ga implementirati, testirati na različitim primjerima i vidjeti kako radi. Greške postaju prilike za učenje -- kada program ne radi kako očekujete, otkrivate nijanse logičkih pravila.

\section*{Curry-Howard izomorfizam: Most između svjetova}

Jedan od najfascinantnijih rezultata 20. stoljeća je otkriće da postoji duboka matematička veza između logičkih dokaza i računalnih programa. Ovu vezu, poznatu kao Curry-Howard izomorfizam, neovisno su otkrili Haskell Curry i William Alvin Howard.

Izomorfizam pokazuje sljedeće ekvivalencije:
\begin{align}
\text{Logičke formule} &\leftrightarrow \text{Tipovi u programiranju}\\
\text{Dokazi} &\leftrightarrow \text{Programi}\\
\text{Normalizacija dokaza} &\leftrightarrow \text{Izvršavanje programa}
\end{align}

Što to znači? Svaki put kada pišete funkciju tipa \texttt{A -> B}, vi zapravo tvrdite da možete dokazati B iz A. Kada pozovete tu funkciju s argumentom tipa A, vi primjenjujete modus ponens!

Razmotrimo konkretan primjer. Logička formula $P \rightarrow (Q \rightarrow P)$ kaže: ``Ako je P istinito, onda bez obzira na Q, P je istinito.'' U Pythonu, ova formula odgovara funkciji:
\begin{verbatim}
def konstanta(p):
    def ignoriraj(q):
        return p
    return ignoriraj
\end{verbatim}

Funkcija prima vrijednost \texttt{p} i vraća funkciju koja ignorira svoj argument i uvijek vraća \texttt{p}. To je dokaz logičke formule pretočen u program!

\section*{Načini mišljenja: Logičko i računalno}

Logičko i računalno mišljenje dijele ključne vještine:

\textbf{Apstrakcija:} I logičari i programeri moraju identificirati bitne strukture i zanemariti nevažne detalje. Kada kažemo ``Svi ljudi su smrtni'', apstrahiramo od individualnih razlika. Kada pišemo funkciju, apstrahiramo od konkretnih vrijednosti.

\textbf{Dekompozicija:} Složene probleme rastavljamo na jednostavnije. U logici, složene formule analiziramo kroz njihove komponente. U programiranju, velike programe gradimo od manjih funkcija.

\textbf{Preciznost:} I logika i programiranje zahtijevaju precizno izražavanje. Dvosmislenost koja je prihvatljiva u svakodnevnom govoru može dovesti do pogrešnih zaključaka ili bugova u kodu.

\textbf{Rekurzija:} Koncept koji je ključan u oba područja. U logici, možemo definirati beskonačne strukture pomoću konačnih pravila. U programiranju, rekurzivne funkcije omogućavaju elegantna rješenja složenih problema.

\section*{Od temelja prema naprijed}

Ovaj udžbenik vodi vas kroz postupno otkrivanje veze između logike i programiranja. Počinjemo s osnovama -- što su sudovi i kako ih kombinirati. Kroz Python implementacije, ove apstraktne koncepte činimo konkretnima i provjerljivima.

Kako napredujete, otkrit ćete da logika nije samo teoretska disciplina. Ona je alat koji omogućava:
\begin{itemize}
\item Jasnije razmišljanje o problemima
\item Preciznije izražavanje ideja
\item Sustavnije pristupanje složenim zadacima
\item Pouzdanije zaključivanje
\end{itemize}

Svaki koncept koji naučite ima direktnu primjenu u programiranju. Razumijevanje logike sudova pomaže u pisanju uvjetnih izraza. Logika predikata ključna je za rad s bazama podataka. Prirodna dedukcija povezana je s tipskim sustavima modernih programskih jezika.

\section*{Poziv na istraživanje}

Logika i programiranje nisu samo alati -- oni su načini razumijevanja svijeta. Kroz ovaj udžbenik, pozivamo vas da istražite kako formalni sustavi mogu opisati i stvarnost i mišljenje o stvarnosti.

Svaka Jupyter bilježnica je laboratorij u kojem možete eksperimentirati. Mijenjajte kod, testirajte granične slučajeve, pokušajte slomiti sustave. Kroz ovo aktivno istraživanje, razvit ćete intuiciju koja je jednako važna kao i formalno znanje.

Zapamtite: u logici, kao i u programiranju, greške su prilike za učenje. Kada program ne radi kako očekujete, ili kada zaključak nije valjan kako ste mislili, vi otkrivate suptilnosti logičkih principa. Ova otkrića su najvrjedniji dio učenja.

Krenimo zajedno na ovo putovanje kroz logiku i kod, gdje svaki novi koncept otvara vrata dubljeg razumijevanja i načina na koji mislimo, zaključujemo i stvaramo.

