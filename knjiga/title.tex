\begin{titlepage}
    \thispagestyle{empty}
    \centering

    % Use \vfill to flexibly manage vertical space and prevent page overflow
    \vspace*{1.5cm} % Pushes the content down from the top margin

    %---------------------------------------------------------------
    % TITLE AND SUBTITLE
    %---------------------------------------------------------------
    {\fontsize{40}{48}\selectfont\bfseries\color{pythonblue}
    LOGIKA U KODU}

    \vspace{0.5cm}

    % A subtle decorative rule
    \rule{0.4\textwidth}{0.4pt}

    \vspace{0.8cm}

    {\Large\color{pythongray}
    Elementi logike kroz programski jezik Python}

    \vfill % Adds flexible space after the title block

    %---------------------------------------------------------------
    % VENN DIAGRAM
    % Recreates the visual from your original PDF
    %---------------------------------------------------------------
    \begin{center}
    \begin{tikzpicture}[scale=1.1, every node/.style={transform shape, text=black!70}]
        % Circles with fill and opacity to show intersections
        \begin{scope}[fill opacity=0.55, line width=1.2pt]
            \fill[logika, draw=logika!80!black] (0, 1.3) circle (2.2cm);
            \fill[filozofija, draw=filozofija!80!black] (-1.1, -0.6) circle (2.2cm);
            \fill[informatika, draw=informatika!80!black] (1.1, -0.6) circle (2.2cm);
        \end{scope}

        % Main labels for each field
        \node at (0, 2.7) {\Large\bfseries LOGIKA};
        \node[anchor=east] at (-2.3, 0.6) {\Large\bfseries FILOZOFIJA};
        \node[anchor=west] at (2.3, 0.6) {\Large\bfseries INFORMATIKA};

        % Symbols inside the main part of each circle
        \node at (0, 0.2) {\huge $\forall \exists$};
        \node at (-1.5, -1.2) {\huge $\Phi$};
        \node at (1.5, -1.2) {\huge \texttt{</>}};

%        % Labels for the intersection areas
%        \node[align=center, font=\small] at (-0.9, 0.5) {formalni\\dokazi};
%        \node[align=center, font=\small] at (0.9, 0.5) {Boolean\\logika};
%        \node[align=center, font=\small] at (0, -1.5) {etika\\AI};
        \node[font=\huge] at (0, -0.6) {λ};
    \end{tikzpicture}
    \end{center}

    \vfill % Adds flexible space between the diagram and the author info

    %---------------------------------------------------------------
    % COURSE AND AUTHOR INFORMATION
    %---------------------------------------------------------------
    {\large\scshape Priručnik za kolegij}\\[0.5cm]
    {\Large\bfseries ,,Logika i programiranje''}\\[1.5cm]
    {\huge Davor Lauc}

    \vfill % Pushes the institution details to the bottom

    %---------------------------------------------------------------
    % INSTITUTION AND YEAR
    %---------------------------------------------------------------
    \begin{center}
        {\large\scshape Filozofski fakultet}\\[0.3cm]
        {\small Sveučilište u Zagrebu}\\[1cm]
        {\huge\bfseries 2025}
    \end{center}

    \vspace*{1cm} % Adds a bit of margin at the very bottom

\end{titlepage}