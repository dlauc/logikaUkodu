
\chapter{Uvod u Python za studente filozofije i ostalih ne-tehničkih grupa)}
\label{chap:Uvod u Pzthon}

\section{Zašto bi se filozof zanimao za programiranje?}

Na prvi pogled, svijet filozofije i svijet programiranja mogu se činiti kao dva potpuno odvojena svemira. Jedan se bavi vječnim pitanjima o smislu, postojanju i vrijednostima, dok se drugi bavi preciznim uputama za strojeve. Međutim, ispod površine, ova dva svijeta dijele duboke i iznenađujuće veze. Logika, temeljni alat filozofske analize, ujedno je i srce svakog računalnog programa. Način na koji strukturiramo argumente, definiramo pojmove i izvodimo zaključke u filozofiji ima svoj odraz u načinu na koji pišemo kod.

Učenje programskog jezika \pyinline{Python}, stoga, za studenta filozofije nije samo stjecanje tehničke vještine, već i prilika za istraživanje poznatih koncepata iz nove perspektive. Kroz \pyinline{Python}, apstraktni pojmovi poput varijabli, uvjeta i petlji postaju konkretni alati s kojima možete raditi, eksperimentirati i stvarati.

Ovo poglavlje je osmišljeno kao blagi uvod u \pyinline{Python}, posebno prilagođen studentima humanističkih i društvenih znanosti. Nećemo se baviti složenim matematičkim problemima niti dubokim tehničkim detaljima. Umjesto toga, fokusirat ćemo se na osnove jezika, koristeći primjere koji su vam bliski: analizu teksta, rad s riječima i rečenicama, te istraživanje ideja kroz kod. Koristit ćemo se \textbf{Jupyter bilježnicama}, interaktivnim okruženjem koje omogućuje pisanje koda, teksta i vizualizacija na jednom mjestu, čineći učenje intuitivnim i zabavnim.

Dok budete prolazili kroz ovo poglavlje, potičem vas da ne gledate na kod samo kao na niz naredbi, već kao na novi način izražavanja i strukturiranja misli. Možda ćete otkriti da vam učenje programiranja može pomoći da postanete precizniji u svom filozofskom promišljanju, jasniji u svom izražavanju i kreativniji u svom pristupu problemima. Dobrodošli u svijet \pyinline{Python}a!

\section{Osnovni pojmovi: Varijable, tipovi podataka i izrazi}
\label{sec:osnovnipojmovi}

\subsection{Varijable: Imenovanje ideja}

U filozofiji, često koristimo simbole ili nazive kako bismo predstavili složene ideje. Na primjer, u logici, slovo \logsud{P} može predstavljati propoziciju "Svi ljudi su smrtni". Na sličan način, u \pyinline{Python}u koristimo \textbf{varijable} kao imenovane spremnike za pohranu podataka.

\begin{definicijaokvir}
    \textbf{Varijabla} je imenovani prostor u memoriji koji služi za pohranu vrijednosti. Ime varijable (identifikator) koristimo kako bismo pristupili pohranjenoj vrijednosti.
\end{definicijaokvir}

Varijablu možete zamisliti kao oznaku koju pridružujete nekoj vrijednosti. Operator dodjele, znak jednakosti (\pyinline{=}), koristi se za dodjeljivanje vrijednosti varijabli.

\begin{primjerokvir}
    Dodjeljivanje vrijednosti varijablama.
    \begin{pythoncode}
pozdrav = "Zdravo, svijete!"
godinarodenjakanta = 1724
pipriblizno = 3.14159
    \end{pythoncode}
    U ovom primjeru, \pyinline{pozdrav}, \pyinline{godinarodenjakanta} i \pyinline{pipriblizno} su nazivi varijabli. Jednom kada definiramo varijablu, možemo je koristiti u daljnjem kodu, na primjer za ispis njezine vrijednosti pomoću ugrađene funkcije \pyinline{print()}.
    \begin{pythoncode}
print(pozdrav)
print(godinarodenjakanta)
    \end{pythoncode}
    \begin{codeoutput}
Zdravo, svijete!
1724
    \end{codeoutput}
\end{primjerokvir}


\subsection{Tipovi podataka: Različite vrste informacija}

U filozofiji, razlikujemo različite vrste pojmova: konkretne, apstraktne, pojedinačne, opće. Slično tome, u \pyinline{Python}u, svaka vrijednost pripada određenom \textbf{tipu podataka}. Osnovni tipovi podataka koje ćemo za početak koristiti su:
\begin{itemize}[leftmargin=*]
    \item \textbf{String (\pyinline{str}):} Niz znakova, odnosno tekstualni podaci. Stringovi se uvijek pišu unutar navodnika (jednostrukih \pyinline{''} ili dvostrukih \pyinline{""}). Primjeri: \pyinline{"Sokrat"}, \pyinline{'Platonova Država'}.
    \item \textbf{Integer (\pyinline{int}):} Cijeli brojevi, bez decimalnog dijela. Primjeri: \pyinline{42}, \pyinline{-399}, \pyinline{2025}.
    \item \textbf{Float (\pyinline{float}):} Brojevi s pomičnim zarezom (decimalni brojevi). Primjeri: \pyinline{3.14}, \pyinline{9.81}, \pyinline{-0.5}.
    \item \textbf{Boolean (\pyinline{bool}):} Logička ili istinitosna vrijednost. Može imati samo dvije vrijednosti: \pyinline{True} (istina) ili \pyinline{False} (laž).
\end{itemize}

\pyinline{Python} je dinamički tipiziran jezik, što znači da ne moramo unaprijed deklarirati tip varijable. Interpretator automatski prepoznaje tip podatka kada dodijelimo vrijednost. Tip varijable možemo provjeriti pomoću ugrađene funkcije \pyinline{type()}.

\begin{primjerokvir}
    Provjera tipova podataka.
    \begin{pythoncode}
filozof = "Aristotel"
godinarodenja = -384
visinaumetrima = 1.7
jeliziv = False

print(type(filozof))
print(type(godinarodenja))
print(type(visinaumetrima))
print(type(jeliziv))
    \end{pythoncode}
    \begin{codeoutput}
<class 'str'>
<class 'int'>
<class 'float'>
<class 'bool'>
    \end{codeoutput}
\end{primjerokvir}

\subsection{Izrazi: Kombiniranje vrijednosti}

U logici, kombiniramo propozicije pomoću veznika ($\konj, \disj, \impl$) kako bismo stvorili složenije izraze. U \pyinline{Python}u, \textbf{izrazi} su kombinacije vrijednosti, varijabli i operatora koje se izračunavaju (evaluiraju) kako bi proizvele novu vrijednost.

\begin{itemize}[leftmargin=*]
    \item \textbf{Aritmetički izrazi:} Koriste standardne matematičke operatore (\pyinline{+}, \pyinline{-}, \pyinline{*}, \pyinline{/}).
    \begin{pythoncode}
a = 10
b = 5
zbroj = a + b
print(zbroj)
    \end{pythoncode}
    \begin{codeoutput}
15
    \end{codeoutput}

    \item \textbf{String izrazi:} Operator \pyinline{+} se može koristiti za spajanje (\textit{konkatenaciju}) stringova.
    \begin{pythoncode}
ime = "Immanuel"
prezime = "Kant"
punoime = ime + " " + prezime
print(punoime)
    \end{pythoncode}
    \begin{codeoutput}
Immanuel Kant
    \end{codeoutput}
\end{itemize}

\section{Strukture podataka: Organiziranje misli}
\label{sec:strukturepodataka}

Dok osnovni tipovi podataka predstavljaju pojedinačne vrijednosti, \textbf{strukture podataka} služe za organiziranje i pohranu više vrijednosti u jednoj varijabli.

\subsection{Liste: Uređeni nizovi argumenata}

U filozofskim tekstovima, često nailazimo na nabrajanja ili nizove ideja, poput Aristotelovih četiriju uzroka. U \pyinline{Python}u, \textbf{liste} su strukture podataka koje nam omogućuju pohranu uređenog niza elemenata. Elementi liste se navode unutar uglatih zagrada \pyinline{[]}, odvojeni zarezima.

\begin{definicijaokvir}
    \textbf{Lista} (\pyinline{list}) je promjenjiva, uređena kolekcija elemenata. "Uređena" znači da elementi zadržavaju redoslijed kojim su dodani. "Promjenjiva" znači da možemo dodavati, uklanjati ili mijenjati elemente nakon što je lista stvorena.
\end{definicijaokvir}

\begin{primjerokvir}
    Kreiranje i pristupanje elementima liste.
    \begin{pythoncode}
aristoteloviuzroci = ["materijalni", "formalni", "djelatni", "svršni"] # Popiy Aristotelovih uzroka

# Pristupanje elementima pomoću indeksa
# Indeksiranje počinje od 0!
prviuzrok = aristoteloviuzroci[0]
treciuzrok = aristoteloviuzroci[2]

print("Prvi uzrok je:", prviuzrok)
print("Treći uzrok je:", treciuzrok)
    \end{pythoncode}

    \begin{codeoutput}
Prvi uzrok je: materijalni
Treći uzrok je: djelatni
    \end{codeoutput}

    Liste su promjenjive. Možemo im dodavati nove elemente metodom \pyinline{append()} ili uklanjati postojeće metodom \pyinline{remove()}.

    \begin{pythoncode}
aristoteloviuzroci.append("imaginarni") # Dodavanje petog, "imaginarnog" uzroka
print(aristoteloviuzroci)

# Uklanjanje "imaginarnog" uzroka
aristoteloviuzroci.remove("imaginarni")
print(aristoteloviuzroci)
    \end{pythoncode}
    \begin{codeoutput}
['materijalni', 'formalni', 'djelatni', 'svršni', 'imaginarni']
['materijalni', 'formalni', 'djelatni', 'svršni']
    \end{codeoutput}
\end{primjerokvir}

\subsection{Rječnici: Asocijativni parovi pojmova i definicija}

U filozofiji, često definiramo pojmove tako da im pridružujemo njihove definicije. U \pyinline{Python}u, \textbf{rječnici} (\pyinline{dict}) omogućuju pohranu podataka u obliku parova \textbf{ključ-vrijednost}. Ključevi su jedinstveni i koriste se za pristup pripadajućim vrijednostima.

\begin{definicijaokvir}
    \textbf{Rječnik} (\pyinline{dict}) je promjenjiva, neuređena kolekcija parova ključ-vrijednost. Svaki ključ mora biti jedinstven unutar rječnika.
\end{definicijaokvir}

Rječnici se definiraju unutar vitičastih zagrada \pyinline{{}}, a parovi ključ-vrijednost odvojeni su dvotočkom.

\begin{primjerokvir}
    Kreiranje i korištenje rječnika.
    \begin{pythoncode}
filozofskirjecnik = {
    "epistemologija": "grana filozofije koja se bavi znanjem",
    "metafizika": "grana filozofije koja se bavi prvim uzrocima i principima bića",
    "etika": "grana filozofije koja se bavi moralom"
}

# Pristupanje vrijednosti pomoću ključa
definicijaetike = filozofskirjecnik["etika"]
print(definicijaetike)

# Dodavanje novog para
filozofskirjecnik["logika"] = "znanost o metodama i principima ispravnog zaključivanja"
print(filozofskirjecnik["logika"])
    \end{pythoncode}
    \begin{codeoutput}
grana filozofije koja se bavi moralom
znanost o metodama i principima ispravnog zaključivanja
    \end{codeoutput}
\end{primjerokvir}

\section{Kontrola toka: Usmjeravanje argumentacije}
\label{sec:kontrolatoka}

Programi se ne izvršavaju uvijek linearno, od prve do zadnje naredbe. \textbf{Kontrola toka} odnosi se na naredbe koje nam omogućuju da usmjeravamo tijek izvršavanja programa, donosimo odluke i ponavljamo operacije.

\subsection{Uvjetno izvršavanje: \pyinline{if}, \pyinline{elif}, \pyinline{else}}

U filozofskoj argumentaciji, često koristimo uvjetne rečenice oblika "Ako $\logsud{P}$, onda $\logsud{Q}$". U \pyinline{Python}u, \textbf{uvjetne naredbe} nam omogućuju da izvršimo određeni dio koda samo ako je zadovoljen neki uvjet. Uvjet je izraz koji se evaluira kao \pyinline{True} ili \pyinline{False}.

\begin{primjerokvir}
    Korištenje \pyinline{if-else} strukture.
    \begin{pythoncode}
tvrdnja = "Sokrat je smrtan"

if "Sokrat" in tvrdnja:
    print("Tvrdnja se odnosi na Sokrata.")
else:
    print("Tvrdnja se ne odnosi na Sokrata.")
    \end{pythoncode}
    \begin{codeoutput}
Tvrdnja se odnosi na Sokrata.
    \end{codeoutput}

    Možemo koristiti i \pyinline{elif} (skraćeno od \textit{else if}) za provjeru više uzastopnih uvjeta.

    \begin{pythoncode}
godina = 1804

if godina < 476:
    print("Antička filozofija")
elif 476 <= godina < 1500:
    print("Srednjovjekovna filozofija")
else:
    print("Moderna i suvremena filozofija")
    \end{codeoutput}
    \begin{codeoutput}
Moderna i suvremena filozofija
    \end{codeoutput}
\end{primjerokvir}

\subsection{Ponavljanje: \pyinline{for} petlja}

Često je potrebno ponoviti istu radnju više puta. Na primjer, analizirati svaku riječ u rečenici. U \pyinline{Python}u, \textbf{\pyinline{for} petlja} nam omogućuje da iteriramo (prolazimo) kroz elemente sekvence (poput liste ili stringa) i za svaki element izvršimo određeni blok koda.

\begin{primjerokvir}
    Iteriranje kroz listu pomoću \pyinline{for} petlje.
    \begin{pythoncode}
stoickevrline = ["mudrost", "pravednost", "hrabrost", "umjerenost"]

print("Prema stoicima, temeljne vrline su:")
for vrlina in stoickevrline:
    print("- " + vrlina)
    \end{pythoncode}
    \begin{codeoutput}
Prema stoicima, temeljne vrline su:
- mudrost
- pravednost
- hrabrost
- umjerenost
    \end{codeoutput}

    U ovom primjeru, varijabla \pyinline{vrlina} se naziva \textit{varijabla petlje}. U svakom prolasku (iteraciji) kroz petlju, ona poprima vrijednost sljedećeg elementa iz liste \pyinline{stoickevrline}.
\end{primjerokvir}

\section{Funkcije: Modularizacija i ponovna upotreba misli}
\label{sec:funkcije}

U filozofiji, kompleksne ideje često razlažemo na manje, razumljivije dijelove. U \pyinline{Python}u, \textbf{funkcije} nam omogućuju da grupiramo niz naredbi u logičku cjelinu koju možemo pozvati više puta. Time se izbjegava ponavljanje koda i programi postaju organiziraniji i lakši za čitanje.

\begin{definicijaokvir}
    \textbf{Funkcija} je imenovani blok koda koji izvršava određeni zadatak. Može primiti ulazne podatke (argumente) i vratiti izlaznu vrijednost.
\end{definicijaokvir}

Funkcije definiramo pomoću ključne riječi \pyinline{def}.

\begin{primjerokvir}
    Definiranje i pozivanje jednostavne funkcije.
    \begin{pythoncode}
def pozdravifilozofa(ime):
    """
    Ova funkcija ispisuje pozdrav filozofu čije je ime
    prolijeđeno kao argument.
    """
    print("Pozdrav, " + ime + "!")

# Pozivanje funkcije
pozdravifilozofa("Platon")
pozdravifilozofa("Nietzsche")
    \end{pythoncode}
    \begin{codeoutput}
Pozdrav, Platon!
Pozdrav, Nietzsche!
    \end{codeoutput}
    Tekst unutar trostrukih navodnika odmah nakon definicije funkcije naziva se \textit{docstring} i služi kao dokumentacija funkcije.
\end{primjerokvir}

Funkcije mogu vraćati vrijednost pomoću naredbe \pyinline{return}. Vraćena vrijednost se tada može pohraniti u varijablu ili koristiti u daljnjim izrazima.

\begin{primjerokvir}
    Funkcija koja vraća vrijednost.
    \begin{pythoncode}
def sastaviime(ime, prezime):
    """Sastavlja puno ime iz dva dijela."""
    return ime + " " + prezime

punoimefilozofa = sastaviime("Simone", "de Beauvoir")
print(punoimefilozofa)
    \end{pythoncode}
    \begin{codeoutput}
Simone de Beauvoir
    \end{codeoutput}
\end{primjerokvir}

\section{Primjer iz prakse: Analiza filozofskog teksta}
\label{sec:primjerprakse}

Sada ćemo primijeniti sve što smo naučili na konkretnom primjeru: analizi kratkog filozofskog teksta. Cilj nam je prebrojati koliko se puta svaka riječ pojavljuje u poznatoj Descartesovoj izreci.

\begin{primjerokvir}
    Brojanje riječi u tekstu.
    \begin{pythoncode}
tekst = "Mislim, dakle jesam. Jesam, dakle postojim." # Korak 1: Definiramo tekst za analizu
print("Originalni tekst:", tekst)

# Korak 2: Priprema teksta
# Pretvaramo sva slova u mala slova kako 'Mislim' i 'mislim' ne bi bili različite riječi
tekstmali = tekst.lower()
# Uklanjamo interpunkcijske znakove
tekstbeztocke = tekstmali.replace('.', '')
tekstcisti = tekstbeztocke.replace(',', '')
print("Očišćeni tekst:", tekstcisti)

# Korak 3: Tokenizacija - razdvajanje teksta u listu riječi
rijeci = tekstcisti.split()
print("Lista riječi:", rijeci)

# Korak 4: Brojanje riječi pomoću rječnika
brojacrijeci = {}
for rijec in rijeci:
    if rijec in brojacrijeci:
        # Ako riječ već postoji u rječniku, povećaj brojač za 1
        brojacrijeci[rijec] = brojacrijeci[rijec] + 1
    else:
        # Ako je ovo prvo pojavljivanje riječi, dodaj je u rječnik s vrijednošću 1
        brojacrijeci[rijec] = 1

# Korak 5: Ispis rezultata
print("\nFrekvencija riječi:")
for rijec, broj in brojacrijeci.items():
    print(f"'{rijec}': {broj}")
    \end{pythoncode}

    \begin{codeoutput}
Originalni tekst: Mislim, dakle jesam. Jesam, dakle postojim.
Očišćeni tekst: mislim dakle jesam jesam dakle postojim
Lista riječi: ['mislim', 'dakle', 'jesam', 'jesam', 'dakle', 'postojim']

Frekvencija riječi:
'mislim': 1
'dakle': 2
'jesam': 2
'postojim': 1
    \end{codeoutput}
    Ovaj primjer integrira varijable, stringove i njihove metode (\pyinline{.lower()}, \pyinline{.replace()}, \pyinline{.split()}), liste, rječnike, \pyinline{for} petlju i \pyinline{if-else} uvjetnu logiku kako bi se riješio konkretan problem iz domene analize teksta.
\end{primjerokvir}

\section{Zaključak: Sljedeći koraci}
\label{sec:zakljucak}

Ovo poglavlje pružilo je kratak pregled osnovnih elemenata programskog jezika \pyinline{Python}. Vidjeli smo kako varijable i tipovi podataka predstavljaju osnovne gradivne blokove, kako strukture podataka poput lista i rječnika organiziraju informacije, kako kontrola toka usmjerava izvršavanje programa i kako funkcije omogućuju modularnost i ponovnu upotrebu koda.

Kao studenti filozofije, sada imate temelj za daljnje istraživanje. Sljedeći koraci mogli bi uključivati:
\begin{itemize}
    \item \textbf{Rad s tekstom:} \pyinline{Python} je izuzetno moćan za analizu teksta. Možete istražiti kako brojati riječi, analizirati sentiment, tražiti određene pojmove u velikim tekstualnim korpusima (npr. djelima pojedinih filozofa) i još mnogo toga. Biblioteke poput \pyinline{NLTK} (Natural Language Toolkit) i \pyinline{spaCy} otvaraju vrata svijeta \textit{računalne lingvistike}.
    \item \textbf{Vizualizacija podataka:} Pomoću biblioteka kao što su \pyinline{Matplotlib} i \pyinline{Seaborn}, možete vizualizirati odnose između pojmova, učestalost riječi ili druge uvide koje dobijete analizom teksta, pretvarajući apstraktne podatke u jasne grafove.
    \item \textbf{Web scraping:} Možete naučiti kako automatski prikupljati tekstualne podatke s web stranica, na primjer, s filozofskih enciklopedija ili online arhiva.
\end{itemize}

Najvažnije je da se ne bojite eksperimentirati. Jupyter bilježnice su idealno okruženje za to. Pokušajte mijenjati primjere, postavljati si vlastite male probleme i tražiti rješenja. Programiranje, kao i filozofija, je vještina koja se razvija kroz praksu, znatiželju i upornost. Sretno s kodiranjem!

\chapter*{Vježbe za poglavlje \ref{chap:uvod}}
\addcontentsline{toc}{chapter}{Vježbe za poglavlje \thechapter}

\begin{vježba}
    Kreirajte rječnik koji sadrži pet vaših omiljenih filozofa kao ključeve, a njihove glavne filozofske ideje ili djela kao vrijednosti. Zatim, koristeći \pyinline{for} petlju, ispišite svakog filozofa i njegovu ideju u formatu: \texttt{Ime Filozofa: Glavna ideja}.
\end{vježba}

\begin{vježba}
    Napišite funkciju pod nazivom \pyinline{brojrijeci} koja prima jedan argument (string) i vraća broj riječi u tom stringu. (Savjet: metoda \pyinline{split()} bi mogla biti korisna). Testirajte funkciju s nekoliko rečenica.
\end{vježba}

\begin{vježba}
    Napišite program koji provjerava pripada li godina određenom filozofskom razdoblju.
    \begin{enumerate}
        \item Definirajte listu koja sadrži nekoliko filozofa egzistencijalizma, npr. \pyinline{egzistencijalisti = ["Sartre", "Camus", "Kierkegaard"]}.
        \item Pitajte korisnika da unese ime filozofa pomoću funkcije \pyinline{input()}.
        \item Koristeći \pyinline{if} naredbu i operator \pyinline{in}, provjerite nalazi li se uneseno ime u vašoj listi.
        \item Ispišite odgovarajuću poruku, npr. \texttt{Sartre je egzistencijalist.} ili \texttt{Platon nije egzistencijalist.}.
    \end{enumerate}
\end{vježba}